\section{S10 -- Systemy wbudowane w strukturach programowalnych}

Przypomnienie -- \textbf{struktury programowalne} to układy elektroniczne, które mają programowalną strukturę, mogą działać jak dowolny układ cyfrowy (o ile zasoby na to zezwalają). Funkcjonalność jest definiowana przez programistę, a nie producenta, co pozwala zmniejszyć koszty, bo z punktu widzenia producenta, produkowane układy są identyczne. PLD są programowane za pomocą języków opisu sprzętu (najpopularniejsze -- Verilog, VHDL), a narzędzie syntezy przetwarza kod programu na fizyczne połączenia. Wyróżnia się układy PLD takie jak: 
\begin{itemize}
\item SPLD -- PLA, PLE, PAL; najprostsze, mają ograniczone możliwości funkcjonalne i logiczne,
\item CPLD -- koncepcyjnie podobne do SPLD, ich architektura opiera się na makrokomórkach, które tworzą bloki funkcyjne, łączone są przez matrycę połączeń,
\item FPGA -- najbardziej zaawansowane, składają się z CLB łączonych w bloki, łączone są przez linie traktów połączeniowych.
\end{itemize}
Więcej informacji można znaleźć w punkcie K9, nie ma się co tu rozpisywać więcej, bo ileż tu można upchnąć ciekawych informacji \texttt{:-)))}.

\textbf{System wbudowany} -- system komputerowy o specjalnym przeznaczeniu, jest integralną częścią obsługiwanego przez niego sprzętu. System ten spełnia wymagania, które zostały określone do zadań, które ma wykonywać. Tak więc zgodnie z tą definicją komputer osobisty nie jest systemem wbudowanym. Systemy wbudowane wykorzystują mikroprocesory bądź mikrokontrolery, które są zaprogramowane do wykonywania tylko i wyłącznie tychże zadań (mają dedykowane oprogramowanie) -- dlatego są nazwane systemami szczególnego przeznaczenia. Systemy te można także oprzeć na strukturach programowalnych. Dzięki wyspecjalizowaniu tych systemów, pobierają one mniej energii, mają mniejsze rozmiary w stosunku do systemów ogólnego przeznaczenia. Systemy te także cechują się wysoką niezawodnością -- są odporne na awarie i zakłócenia.

Z systemami wbudowanymi można się spotkać na każdym kroku. Występują one m.in. w:
\begin{itemize}
\item komputerach pokładowych (w aucie, samolocie, etc.), 
\item systemach naprowadzania rakiet (\texttt{;-)}),
\item sprzęcie medyczny,
\item bankomatach,
\item ,,yntelygentnych'' urządzeniach (telewizory, lodówki, boję się pomyśleć co jeszcze),
\item routerach,
\item systemach alarmowych,
\item drukarkach,
\item odtwarzaczach DVD/Blu-Ray/etc.,
\item można tak wymieniać do wieczora, ale chyba komisji to się zbytnio nie spodoba.
\end{itemize} 

System wbudowany składa się z trzech głównych części:
\begin{itemize}
\item obwody wejściowe (czujniki, przetworniki A/C lub C/A, etc.) -- zbierają dane,
\item jednostka centralna -- pobiera dane, przetwarza je przez swój algorytm oraz przekazuje na obwody wyjściowe,
\item obwody wyjściowe -- mogą to być porty komunikacyjne, czy też wyjścia sterujące.
\end{itemize}

Realizacja systemów wbudowanych na strukturach programowalnych może się odbyć na CPLD lub FPGA (SPLD nie są wystarczająco duże).

Układy CPLD oraz FPGA można zaprogramować programowymi procesorami. Jednym z nich, dosyć popularnym jest PicoBlaze, (8-bitowy). Jest on stworzony w HDLu. Posiada on cechy podobne do istniejących na rynku mikrokontrolerach (takich jak: 8051, ATtiny13, ST6215). W odróżnieniu od nich, pozwala na dowolną liczbę peryferii, ograniczeniem jest liczba bloków wybranego PLD.

Istnieją także inne mikroprocesory, którymi można zaprogramować układy PLD, jednakże wymagają one FPGA (nie CPLD). Są to: MicroBlaze, OpenRISC, ARM Cortex-M1 (oczywiście, jest ich więcej). MicroBlaze jest podobny do PicoBlaze, jednak jest to procesor programowy 32-bitowy, posiada większą liczbę zasobów, oraz wyposażony w jednostkę FPU (czyli obsługuje nasze ulubione liczby zmiennoprzecinkowe). Z ciekawostek, obsługa MicroBlaze jest zawarta w kodzie źródłowym jądra Linuksa.

Programowanie takich procesorów programowych odbywa się w dwóch etapach. Pierwszy z nich polega na definicji układu przy pomocy HDLa -- tutaj jest ustalana ,,budowa'' procesora. Następnie jest to przekształcane na sprzętową implementację jako układy kombinacyjne lub sekwencyjne. Kolejny krok to stworzenie kodu programu w Asemblerze bądź C/C++, a następnie zaprogramowanie procesora.

Łączenie mikroprocesorów i układów programowalnych może się odbyć na cztery sposoby:
\begin{enumerate}
\item \textbf{Mikroprocesor/mikrokontroler}

Połączenie mikroprocesora z peryferiami w jeden lub kilka układów scalonych.
\item \textbf{$\mu$procesor/$\mu$kontroler + PLD}

Rozwiązanie to pozwala rozszerzać systemy procesorowe o dodatkowe możliwości, które dają układy programowalne.
\item \textbf{PLD zintegrowane z $\mu$procesorem / inaczej: $\mu$kontroler z wbudowaną częścią FPGA w jednym układzie scalonym}

Układy PLD z wbudowanym procesorem, np. Xilinx Virtex II + PowerPC. Rozwiązanie to nie przyjęło się ze względu na wysoką cenę.

\item \textbf{Układy PLD z zaimplementowanym $\mu$procesorem}

Rozwiązanie używane do projektowania i weryfikacji projektów przez firmy produkujące układy scalone. Wraz ze wzrostem wydajności i pojemności tańszych układów ta konfiguracja zaczyna być konkurencyjna dla mikrokontrolerów. Rozwiązanie to wydaje się być optymalne dla sprzętowych procesorów (PicoBlaze, MicroBlaze, etc.).
\end{enumerate}

Układy programowalne dają elastyczność, struktura systemu wbudowanego o taki układ może być dostosowana ściśle do potrzeb oraz może być w prosty sposób zmieniona. Gotowe mikroprocesory nie pozwalają na manipulację liczbą portów czy możliwych peryferii do podłączenia. Można zbudować system oparty o wiele procesorów, gdzie każdy ma przydzielone inne zadanie do wykonania. 

Wadą takiego rozwiązania jest większy pobór mocy.

Układy programowalne pozwalają na zrealizowanie dowolnych peryferii wewnątrz układu, np. układy DVB, szyfrujące, bloki przetwarzania sygnałów, porty szeregowe i równoległe, magistrale, etc.
