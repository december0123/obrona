\section{S8 -- Spójność sieciowego systemu operacyjnego}

%nie ma spójności, tylko halucynacje i śmierć z niedożywienia

\textbf{Sieciowy system operacyjny (NOS -- Network Operating System)} -- umożliwia komunikacje miedzy wieloma urządzeniami a także zasobami sieci. Sieciowy system operacyjny tworzy środowisko, w którym użytkownicy maszyn mają dostęp zasobów sieciowych. Ma on zastosowanie zarówno na serwerach jak i stacjach roboczych.

Do najważniejszych funkcji sieciowego systemu operacyjnego jest zapewnienie komunikacji, poprzez wykorzystanie protokołów komunikacyjnych takich jak:
\begin{itemize}
    \item TCP/IP - sieć Internet
    \item IPX/SPX - sieć Novell 
\end{itemize}

Popularne sieciowe systemy dostępne na rynku to:
\begin{itemize}
    \item Windows Server 2012 R2
    \item NetWare
    \item Unix
    \item Linux
\end{itemize}

Sieciowy system operacyjny można opisywać na podstawie systemu UNIX oraz zgodnych. 

Cechy systemu unixowego:
\begin{itemize}
    \item wieloużytkownikowy,
    \item wielozadaniowy,
    \item podział czasu,
    \item wywłaszczanie procesów,
    \item podsystem plików,
    \item zarządzanie pamięcią,
    \item biblioteki systemowe,
    \item urządzenia dostępne jako pliki specjalne.
\end{itemize}

System powinien posiadać interfejsy programistyczne zgodne z ANSI/C lub Posix:
\begin{itemize}
    \item Procesy tworzone przez funkcję fork(), kończone przez wait() i exit().
    \item Komunikacja między procesami przy użyciu sygnałów.
    \item Obsługa strumieni FIFO i pipe.
    \item Komunikacja przy użyciu pamięci dzielonej.
\end{itemize}

Komunikacja sieciowa w oparciu o sieć magistrali, pierścienia lub gwiazdy. Obsługa komunikacji
zgodnie z modelem ISO/OSI (siedem warstw) przy użyciu TCP/IP. Większość komunikacji odbywa
się w modelu klient-serwer. 

Ważnym z punktu widzenia spójności jest system plików używany w takim systemie. Zakłada on model rozproszony -- DFS (Distributed File System), który przechowuje pliki na wielu serwerach.

Zalety DFS to:
\begin{itemize}
    \item Zwielokrotnienie -- (z ang. replication) wiele kopii danych.
    \item Niezawodność -- wielokrotność egzemplarzy może uchronić przed ich zaginięciem.
    \item Efektywność -- szybszy i łatwiejszy dostęp do danych.
\end{itemize}

Wady DFS:
\begin{itemize}
    \item Problemy ze spójnością.
    \item Edycja danych na jednym serwerze nie gwarantuje ich zmiany na pozostałych.
    \item Konieczny jest mechanizm zapewniający aktualizację na pozostałych serwerach.
    \item Optymalne rozmieszczenie serwerów sprzyja spójności danych.
\end{itemize}

Sposoby dostępu do danych w systemach rozproszonych:

\textbf{Dostęp zdalny:}
\begin{itemize}
    \item Współdzielony plik zlokalizowany jest na serwerze.
    \item Regularne przesyłanie komunikatów pomiędzy klientem a serwerem -- możliwe opóźnienia.
    \item Prosty sposób, brak problemów ze spójnością.
\end{itemize}

\textbf{Relokacja:}
\begin{itemize}
    \item Przeniesienie pliku do pamięci lokalnej hosta.
    \item Zmniejszenie ruchu w sieci.
    \item Pozostałe hosty odwołują się do nowej lokalizacji pliku.
\end{itemize}

\textbf{Powielanie:}
\begin{itemize}
    \item Kopiowanie pliku do pamięci lokalnej klienta.
    \item Lokalne odwołanie do pliku -- pełna prędkość.
    \item Potencjalna niespójność danych.
\end{itemize}

Metody propagowania zmian do serwera:
\begin{itemize}
    \item Natychmiastowe przepisywanie.
    \item Opóźnione przepisywanie.
    \begin{itemize}
        \item Zastosowanie polityki aktualizacji.
        \item Okresowa aktualizacja.
        \item Aktualizowanie przy zamykaniu pliku.
    \end{itemize}
\end{itemize}

Walidacja danych pomiędzy serwerem a klientem może zostać zrealizowana na 2 sposoby. 
Pierwszy z nich polega na zainicjalizowaniu tego procesu przez klienta:
\begin{itemize}
    \item Przy każdym dostępie do danych lub co stały okres czasu.
    \item Klient wysyła dużo komunikatów do serwera -- skutkuje zwiększeniem się ruchu sieciowego.
    \item Możliwe obciążenia serwera.
\end{itemize}

Drugi sposób polega na zainicjalizowaniu procesu walidacji przez serwer:
\begin{itemize}
    \item Serwer monitoruje stan wszystkich plików.
    \item Klient informuje o zmianach w plikach.
    \item Wszystkie kopie można odczytywać w tym samym czasie.
    \item Tylko jedna kopia może być modyfikowalna.
\end{itemize}

Przykłady sieciowych systemów plików:
\begin{itemize}
    \item NFS -- protokół oparty na UDP lub TCP. Jest standardowym sieciowym systemem plików w systemach uniksowych. Z NFS wiąże się wiele problemów -– przede wszystkim bardzo trudno zapewnić, że dana operacja została wykonana.
    \item SMB --  protokół służący udostępnianiu zasobów komputerowych, m.in. drukarek czy plików. SMB jest protokołem typu klient-serwer, a więc opiera się na systemie zapytań generowanych przez klienta i odpowiedzi od serwera. 
    \item Microsoft DFS -- implementacja rozproszonego systemu plików firmy Microsoft. Do wymiany danych korzysta on z protokołu SMB. DFS był już obecny w Windows NT 4.0, obecnie jest dostępny też w Windows Server 2000/2003/2008 oraz Samba 3.0.
\end{itemize}