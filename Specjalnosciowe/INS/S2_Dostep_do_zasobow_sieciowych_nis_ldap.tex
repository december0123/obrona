\section{S2 -- Mechanizmy  zdalnego  dostępu  do  zasobów  sieciowych (dyski  sieciowe,  mapowanie  uprawnień  dostępu,  sieciowe zarządzania użytkownikami NIS/LDAP) }

\subsection{NFS}
Uniksowy system plików wykorzystujący protokoły TCP, UDP lub SCTP (Stream Control Transmission Protocol). Jest otwartym standardem RFC.
Typowa implementacja:
\begin{enumerate}
	\item Aplikacja serwera \texttt{nfsd} nasłuchuje na porcie 2049 (port standardowy). Konfiguracja pobierana jest z pliku /etc/exports (lista udziałów, uprawnienia, ustawienia połączenia - zapis synchroniczny lub asynchroniczny). Program \texttt{exportfs} umożliwia zarządzanie tabelą udostępnianych katalogów.
	\item Klient uzyskuje dostęp do katalogów za pomocą polecenia mount (serwer jest odpytywany poprzez rpcbind na jakim porcie działa nfsd)
\end{enumerate}

Główne zmiany w kolejnych wersjach:
\begin{enumerate}
	\item NFSv1 - przeznaczony tylko do eksperymentalnego wykorzystania
	\item NFSv2 (RFC 1094, March 1989) - wykorzystywał protokół UDP, ograniczenie rozmiaru plików do 2GB, ze względu na 32 reprezentacje rozmiaru plików.
	\item NFSv3 (RFC 1813, June 1995) - wykorzystanie liczb 64 bitowych do przechowywania rozmiarów plików, zwiększona wydajność ze względu na wsparcie dla zapisu asynchronicznego (brak oczekiwania na potwierdzenie zapisu na dysku po stronie serwera), 
	\item NFSv4 (RFC 3010, 2000; zmodyfikowany RFC 3530, 2003) - wykorzystanie kerberosa do uwierzytelniania użytkowników. Dodano wsparcie dla ACL w modelu podobnym dla windowsa. Możliwości są większe niż w klasycznej implementacji POSIX. Każdna reguła POSIX ma odzwierciedlenie w NFSv4 ACL - domyślnie są one w locie tłumaczone. Oczywiście nie wszystkie reguły NFSv4 ACL zostaną obsłużone przez obsługujący serwer wspierający tylko POSIX ACL.
	\item NFSv4.1 (RFC 5661, January 2010) - dodano mechanizmy równoległego dostępu do zasobów na wielu serwerach - rozszerzenie pNFS.
\end{enumerate}

NFSv4 jest zaimplementowany z myślą o uwierzytelnianiu poszczególnych użytkowników. NFSv2 i NFSv3 weryfikowały tylko komputer!

\subsection{NFSv4 mapowanie uprawnień dostępu}
Jak wspomniano wcześniej NFSv4 obsługuje reguły ACL, które wykraczają poza możliwości POSIX ACL. Dlatego systemy wspierające POSIX muszą dokonać odpowiedniego mapowania.

\begin{enumerate}
	\item mapowanie dokładne - dokonywane w locie przez klienta i serwer linuksowy. Kończy się niepowodzeniem jeśli reguła POSIX nie jest całkowicie zgodna z NFSv4 ACL. Wadami jest duże skomplikowanie wygenerowanych reguł NFSv4 ACL oraz mały zakres wspieranych reguł przez linuksowy serwer przy mapowaniu odwrotnym.
	\item mapowanie stratne
\end{enumerate}

\subsection{SMB}
Rozwija się jako Server Message Block. Początkowo zaprojektowany przez IBM, a obecnie rozwijany przez Microsoft. Znany także pod nazwą CIFS (Common Internet File System). Powszechnie wykorzystywany przez systemy z rodziny Windows. Może być uruchomiony w oparciu o TCP (port 445) lub NetBIOS - Network Basic Input / Output Protocol, czyli działający w warstwie sesji modelu OSI interfejs łączenia się aplikacji na komputerach w tej samej sieci lokalnej (porty UDP 137, 138  oraz TCP 137 i 139).

Od Windows 8 (SMB 3.0) wspiera szyfrowanie transmisji.

Istnieje otwarta implementacja utworzona w oparciu o inżynierię wsteczną - Samba.

\subsection{DFS}
Rozproszony system plików - Distributed File System

Dane przechowywane są na wielu maszynach jednocześnie przy zintegrowaniu lokalizacji logicznej (ładnie ta wikipedia prawi... czyż nie?). W każdym razie użytkownik końcowy widzi wszystko w jednym miejscu, tyle że np. w wielu katalogach. Często stosowana jest replikacja danych pomiędzy maszynami, co dodatkowo zwiększa niezawodność systemu plików.

Rozwiązanie takie pozwala na:
\begin{enumerate}
	\item zmniejszenie ryzyka utraty danych i zwiększenie ich dostępności w przypadku awarii jednego serwera
	\item podział obciążenia pomiędzy serwerami
	\item dla dużych sieci pozwala kierować użytkowników do najbliższego serwera - tzn. tego w Polsce, a nie w Indiach
\end{enumerate}

\subsubsection{Przykładowe implementacje}
\textbf{Microsoft DFS}
Implementacja rozwijana od czasów Windows NT 4.0. Pozwala organizacji na połączenie wielu udziałów SMB w jeden rozproszony system plików. Dla zwiększenia niezawodności stosowana jest przejrzystość lokalizacji oraz nadmiarowość. Możliwa jest konfiguracja w dwóch trybach:
\begin{itemize}
\item Standalone DFS - struktura przechowywana i dostępna tylko na jednym komputerze. Obecnie rzadko spotykany.
\item Domain Based DFS - opiera się o usługę Active Directory. Możliwa jest replikacja danych pomiędzy serwerami.
\end{itemize}

\textbf{GlusterFS}
Otwarto źródłowy rozproszony system plików. Początkowo rozwijany przez firmę Gluster, następnie przez Red Hat po wykupieniu tej pierwszej w 2011 roku. Automatyczna redundancja, możliwy jest dostęp za pomocą klientów NFS, SMB lub natywnego klienta.