\section{S2 -- Mechanizmy  zdalnego  dostępu  do  zasobów  sieciowych (dyski  sieciowe,  mapowanie  uprawnień  dostępu,  sieciowe zarządzania użytkownikami NIS/LDAP) }

\subsection{NFS}

\subsection{SMB}

\subsection{DFS}
Rozproszony system plików - Distributed File System

Dane przechowywane są na wielu maszynach jednocześnie przy zintegrowaniu lokalizacji logicznej (ładnie ta wikipedia prawi... czyż nie?). W każdym razie użytkownik końcowy widzi wszystko w jednym miejscu, tyle że np. w wielu katalogach. Często stosowana jest replikacja danych pomiędzy maszynami, co dodatkowo zwiększa niezawodność systemu plików.

Rozwiązanie takie pozwala na:
\begin{enumerate}
	\item zmniejszenie ryzyka utraty danych i zwiększenie ich dostępności w przypadku awarii jednego serwera
	\item podział obciążenia pomiędzy serwerami
	\item dla dużych sieci pozwala kierować użytkowników do najbliższego serwera - tzn. tego w Polsce, a nie w Indiach
\end{enumerate}

\subsubsection{Microsoft DFS}
Implementacja rozwijana od czasów Windows NT 4.0. Pozwala organizacji na połączenie wielu udziałów SMB w jeden rozproszony system plików. Dla zwiększenia niezawodności stosowana jest przejrzystość lokalizacji oraz nadmiarowość. Możliwa jest konfiguracja w dwóch trybach:
\begin{itemize}
\item Standalone DFS - struktura przechowywana i dostępna tylko na jednym komputerze. Obecnie rzadko spotykany.
\item Domain Based DFS - opiera się o usługę Active Directory. Możliwa jest replikacja danych pomiędzy serwerami.
\end{itemize}

\subsubsection{GlusterFS}
Otwarto źródłowy rozproszony system plików. Początkowo rozwijany przez firmę Gluster, następnie przez Red Hat po wykupieniu tej pierwszej w 2011 roku. Automatyczna redundancja, możliwy jest dostęp za pomocą klientów NFS, SMB lub natywnego klienta.