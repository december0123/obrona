\section{S8 -- Sposoby budowy i zarządzania aplikacjami rozproszonymi za pomocą pakietów należących do standardowej dystrybucji Javy}

Aplikacje rozproszone to takie, które są wykonywane na wielu maszynach celem rozłożenia obciążenia.

Takie aplikacje można zrealizować za pomocą zdalnego wywoływania metod obiektów, bądź przesyłania komunikatów.

\subsection{Gniazda}
Przesyłanie komunikatów wiąże się z wykorzystaniem gniazd sieciowych, a budowa aplikacji opiera się na architekturze klient-serwer.
Samą komunikację można podzielić na strumieniową (TCP) oraz datagramową (UDP).

Wygląda to tak, że zarówno klient jak i serwer otwierają gniazda sieciowe.
Następnie serwer oczekuje na nadejście nowych połączeń.
Klient podaje z jakim adresem IP chce się połączyć, a konkretna usługa jest rozpoznawana po numerze portu.
Gdy serwer odbierze połączenie, otwiera nowe gniazdo, przez które jest kontynuowana wymiana, a stare gniazdo istnieje w celach odbierania nowych połączeń.

\subsection{RMI}
Innym sposobem realizacji takiej aplikacji, który poznaliśmy w czasie studiów jest RMI.
RMI dostarcza mechanizmy pozwalające na udostępnianie oraz wywoływanie metod zdalnych obiektów.

Obiekty, których metody będą wywoływane zdalnie, przede wszystkim muszą implementować pewien konkretny interfejs.
Następnie, namiastka tych obiektów musi zostać zarejestrowana w rejestrze RMI.
Będzie ona służyła do wywoływania ich metod.

Kiedy klient chce wywołać metodę zdalnie, komunikuje się z rejestrem RMI i pobiera informacje o namiastkach i ich metodach.
Po pobraniu namiastek, może on działać na nich tak jakby to były zwyczajne obiekty lokalne -- komunikacja jest więc dla użytkownika niewidoczna.