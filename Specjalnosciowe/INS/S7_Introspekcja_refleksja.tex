\section{S7 -- Różnice pomiędzy introspekcją i odzwierciedleniem – metodami stosowanymi do rozpoznania własności klas lub zmodyfikowania zachowania się aplikacji działających na wirtualnej maszynie Java}

Języki wysokiego poziomu umożliwiają tworzenie programów, które są w stanie w \textbf{czasie wykonania} dokonać inspekcji i modyfikacji innych programów, bądź też samych siebie.

Pojęcia introspekcji i refleksji (odzwierciedlenia) nie są unikalne dla języka Java, jednak właśnie na tym języku się skupię (taka natura pytania).

Są to pojęcia ściśle ze sobą powiązane, jednak nie są tym samym.

\subsection{Introspekcja}

Introspekcja umożliwia programowi inspekcję kodu programów, które już się wykonują (w tym też samego siebie).
Dzięki niej, możemy uzyskać informacje o używanych typach i zdefiniowanych metodach.

Przykładem ze środowiska Javy, gdzie jest używana introspekcja, mogą być narzędzia służące do graficznego projektowania interfejsu użytkownika.

Korzystają one z mechanizmu introspekcji do prześwietlania kodu i wydobywania informacji z ziarenek Javy (JavaBeans), aby z nich skorzystać.

\subsection{Refleksja}

Refleksja korzysta z mechanizmu introspekcji i idzie o krok dalej.
Oprócz samego wydobywania informacji z programu, refleksja umożliwia modyfikowanie go.

Możemy więc wydobyć informacje z jakiegoś obiektu, o którym nic nie wiemy, a następnie wybrać, które metody wywołać i w jaki sposób.

Przykładem, gdzie przydaje się refleksja może być testowanie.
Za pomocą mechanizmu refleksji, framework JUnit wyszukuje metody oznaczone za pomocą adnotacji @Test i w ten sposób wie, które metody służą do testowania i może je wywołać.

W odpowiednich rękach jest to potężne narzędzie, które umożliwia nam pracę z kodem, który mógł jeszcze nie zostać nawet napisany.
Należy jednak pamiętać, że ze względu na to, że wszystko dzieje się w czasie wykonania, nadmierne wykorzystywanie mechanizmu refleksji może negatywnie wpłynąć na wydajność programu.
