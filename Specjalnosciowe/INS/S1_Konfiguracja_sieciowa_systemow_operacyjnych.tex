\section{S1 -- Konfiguracja sieciowa systemów operacyjnych (sterowniki urządzeń sieciowych, ustawienia parametrów sieci lokalnej i TPC, automatyzacja konfiguracji)}

Sieciowy system operacyjny to system, którego jądro ma zintegrowany stos sieciowy.
Mądrze brzmi, ale co to znaczy?
Znaczy to tyle, że jądro takiego systemu posiada
\begin{itemize}
	\item{sterowniki obsługujące karty sieciowe,}
	\item{oprogramowanie obsługi co najmniej trzech najniższych warstw, tj. łącza danych, sieciowej i transportowej}
	\item{funkcje biblioteczne wspierające gniazdka Berkeley, Windows bądź interfejs TLI.}
\end{itemize}

\subsection{Urządzenia}
W systemie możemy wyróżnić między innymi następujące trzy rodzaje urządzeń.

\begin{itemize}
	\item{Znakowe -- są to urządzenia, które obsługują odczyt i zapis znak po znaku. Nie są buforowane i interakcja jest natychmiastowa. Przykład: drukarka.}
	\item{Blokowe -- urządzenia te umożliwiają odczyt i zapis całych bloków. Są buforowane i umożliwiają dostęp do danych w nich przechowywanych. Przykład: dysk, plik.}
	\item{Interfejsy sieciowe -- zamiast znaków czy bloków, obsługują pakiety.}
\end{itemize}

Każde z tych urządzeń jest obsługiwane przez odpowiedni sterownik -- część jądra systemu, odpowiedzialną za obsługę urządzeń.
Sterowniki można dołączać do jądra \textbf{statycznie -- } w trakcje kompilacji -- oraz \textbf{dynamicznie -- } w formie ładowalnych modułów.

Interfejsy można konfigurować lokalnie, za pomocą plików konfiguracyjnych i narzędzi typu ifconfig.
Jednak istnieją również rozwiązania pozwalające na automatyczną konfigurację systemu w momencie jego uruchomienia i włączenia do sieci.

\subsection{BOOTP}
Jest to protokół służący do konfiguracji komputerów.
Taki komputer po uruchomieniu rozsyła na adres rozgłoszeniowy zapytanie BOOTP.
Kiedy serwer odbierze to zapytanie, wysyła klientowi odpowiedź BOOTREPLY zawierającą konfigurację.

Tego protokołu używano również do pobierania obrazu systemu operacyjnego.
Pozwalało to na używanie prostych terminali, które nie posiadały własnych dysków, a całość konfiguracji ściągały z sieci i ładowały do pamięci operacyjnej.

\subsection{DHCP -- Dynamic Host Configuration Protocol}
Następcą BOOTP, jest protokół DHCP.
W odróżnieniu do BOOTP, protokół ten jest przeznaczony dla maszyn posiadające własne dyski i skupia się na automatycznej konfiguracji komputerów, które często mogą zmieniać swoją fizyczną lokalizację.
Potrafi komunikować się z komputerem również po uruchomieniu się systemu, a nie tylko przed, dzięki czemu łatwiejsze staje się odnawianie dzierżawy adresu -- komputera nie trzeba uruchamiać ponownie.

\subsection{PXE -- Preboot eXecution Environment}
Protokół ten jest następcą BOOTP jeśli chodzi o bootowanie systemu z sieci.
Jest on zbudowany na bazie DHCP (do uzyskania adresu i maski) oraz TFTP (do transferu danych).
Gdy komputer uzyska konfigurację sieciową, zaczyna odpytywać sieć w poszukiwaniu serwera TFTP.
Gdy go znajdzie to pobiera z niego obraz systemu.
Protokół ten jest aktualnie standardem pozwalającym na szybkie i proste skonfigurowanie komputerów, np. pracowników dużych firm, których komputery powinny posiadać pewną określoną polityką firmy konfigurację.

Ze względu na konieczność bycia łatwymi do zaimplementowania, podane wyżej protokoły opierają się na UDP, a nie TCP.