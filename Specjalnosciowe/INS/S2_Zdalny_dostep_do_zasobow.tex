\section{S2 -- Mechanizmy zdalnego dostępu do zasobów sieciowych (dyski sieciowe, mapowanie uprawnień dostępu, sieciowe zarządzania użytkownikami NIS/LDAP)}

Udostępnianie zasobów w sieci jest bardzo wygodnym rozwiązaniem.
Zwalnia nas z potrzeby wymieniania się nośnikami danych i kopiowania danych między komputerami.
Rozwiązań jest wiele i od naszych wymagań czy możliwości zależy to, które wybierzemy.

\subsection{FTP}
Jedną z najbardziej rozpowszechnionych usług tego typu są serwery FTP oraz ich bezpieczne odpowiedniki SFTP (tunelowanie ssh) oraz FTPS (ssl/tls).

FTP (File Transfer Protocol) jest to usługa, która pozwala nam na udostępnienie miejsca na serwerze, z którego ludzie mogą pobierać lub wysyłać dane.
Może ona działać w dwóch trybach, z których oba potrzebują dwóch połączeń -- do przesyłania poleceń oraz danych.
Tryby te to
\begin{itemize}
	\item{aktywny -- gdy klient łączy się do serwera i wymienia polecenia, a dopiero później ustanawia nowe połączenie na porcie N+1 w celu przesłania danych,}
	\item{pasywny -- gdy klient od samego początku ustanawia oba połączenia.}
\end{itemize}

\subsection{NAS}
Będąc w temacie zasobów sieciowych nie można oczywiście pominąć dysków sieciowych.
Są to urządzenia wpinane do sieci -- zazwyczaj kablem ethernetowym -- lecz dla użytkowników są widoczne jak zasoby lokalne.
Daje nam to możliwość udostępnienia zasobów raz -- nie musimy wymieniać się nośnikami i kopiować danych.

\subsection{NFS}
Jednak nie trzeba w celach udostępniania danych w sieci kupować dedykowanych urządzeń. 
Można skorzystać z protokołu NFS, który umożliwia udostępnienie swoich danych w sieci.
Udostępnione pliki można zamontować w swoim systemie i operować jak na zasobach lokalnych.

\subsection{SMB}
Innym protokołem jest SMB.

\subsection{Uprawnienia}
NFS musi oczywiście sprawdzić, czy mamy uprawnienia dostępu do zasobów, co sprowadza nas do tematu uprawnień i zarządzania użytkownikami.
W przypadku NFSa, system sprawdza czy nasz użytkownik i jego uprawnienia zgadzają się z tym co jest na serwerze -- jeśli spróbujemy odczytać coś do czego nasz użytkownik nie ma dostępu to się nam to nie uda.
Konfigurację przeprowadza się za pomocą pliku /etc/exports.

\subsection{Zarządzanie użytkownikami}
Dla zarządzania użytkownikami również istnieją różne rozwiązania, m.in. \textit{NIS, LDAP} czy \textit{Active Directory.}

NIS (Yellow Pages) jest protokołem umożliwiającym współdzielenie informacji o użytkownikach.
Działa on w architekturze serwer-klient, gdzie serwer posiada bazę z informacjami na temat użytkowników, a klient z tej bazy może skorzystać.
Można więc tak ustawić sieć, że gdy użytkownik próbuje się zalogować do komputera, a w lokalnym systemie nie widzimy takiego użytkownika, komputer odpytuje serwer z bazą danych i na tej podstawie weryfikuje czy użytkownik może uzyskać dostęp.

NIS ma jednak tę wadę, że klient może pobrać całą bazę, a następnie przeprowadzić atak siłowy celem złamania haseł.

Następnikiem NISa jest LDAP.
LDAP podobnie jak NIS działa w architekturze serwer-klient, jednak różnica polega na tym, że weryfikacja odbywa się po stronie serwera.

Przykładem, gdzie takie rozwiązanie jest wdrożone może być sama politechnika.
Studenci posiadają konta na poczcie, którego można użyć do zalogowania się do sieci Eduroam.
Takie rozwiązanie jest również wdrożone w niektórych laboratoriach.