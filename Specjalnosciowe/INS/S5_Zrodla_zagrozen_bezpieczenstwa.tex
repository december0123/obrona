\section{S5 -- Źródła zagrożeń bezpieczeństwa systemów i usług informatycznych}

\subsection{Natura}
Jednym z zagrożeń dla systemów, które istnieje, a nie zawsze się o nim myśli jest natura.
Woda może zalać sprzęt, piorun uszkodzić instalacje, a zwierzątka poprzegryzać kable i należy się przed tym zabezpieczać.

\subsection{Użytkownicy}
Jednak głównym źródłem zagrożeń dla systemów są ludzie.
Nawet jeśli zaprojektujemy system, który w naszej opinii jest wyjątkowo dobrze zabezpieczony, to ktoś zawsze znajdzie sposób, by nam udowodnić, że się mylimy (czy to przez przypadek, czy umyślnie).

Użytkownicy ustawiają słabe hasła, które łatwo złamać, bądź też udostępniają je na lewo i prawo.
Robią coś i nie pomyślą nawet, że ich działania mogą przyczynić się do ujawnienia tajnych danych czy zakażenia komputera wirusem.

\subsection{Socjotechniki}
Ludzie są łatwowierni.
Jeśli wejdziemy do jakiegoś miejsca odpowiednio ubrani i będziemy sprawiać wrażenie, że dokładnie wiemy co robimy to nikt nas nie zatrzyma.
Można więc w okolice firmy podrzucić odpowiednio zakażony pendrive lub wysłać wiadomość podając się za przedstawiciela banku i poprosić o udostępnienie danych.

\subsection{Haksy}
Również ludzie w postaci programistów są odpowiedzialni za błędy w oprogramowaniu, które mogą pozwalać na nieautoryzowany dostęp do systemu.
Programiści mogą chcieć ułatwić sobie pracę tworząc tak zwane \textit{backdoory} i później ich nie usuwają (ponownie, specjalnie bądź przez przypadek).

Oczywiście poza przyczynami losowymi i czynionymi nieumyślnie mamy też do czynienia z niebezpieczeństwem ze strony ludzi, którzy umyślnie chcą spowodować szkody.
Do zagrożeń ze strony takich osób należą

\subsubsection{Sniffery}
Oprogramowanie podsłuchujące dane przepływające w sieci korzystając z faktu, że niestety internet był projektowany w czasach, gdy bezpieczeństwo w sieci nie było wielkim tematem i dane w niższych warstwach są przesyłane w postaci jawnej.
Wykrywanie ich polega na wykorzystaniu zasady działania snifferów -- wykorzystują one kartę sieciową przełączoną w tryb \textit{promiscuous,} aby przechwytywać pakiety, które nie są do nich adresowane.

Można więc taką kartę wykryć wysyłając do podejrzanego hosta pakiet ICMP ECHO z jego adresem IP, ale cudzym adresem MAC.
Praworządna karta nie powinna takiego pakietu przyjąć, lecz karta podsłuchująca wyśle odpowiedź i w ten sposób można poznać kto jest kto.

\subsubsection{Denial of Service}
Atak typu DoS polega na zajęciu zasobów systemowych (czas CPU, miejsce na dysku) w taki sposób, że inne procesy i użytkownicy nie są w stanie z nich korzystać.
Atak taki można również przeprowadzić z sieci, np. poprzez zasypywanie serwera pakietami.

Walczyć z takim atakiem można poprzez ustawienie limitów w systemie, np. limit na miejsce przeznaczone dla jednego użytkownika, czy liczbę procesów mogących operować jednocześnie.

\subsubsection{Spoofing}
Podszywanie się pod kogoś innego.
\begin{itemize}
	\item{DNS -- wysłanie do serwera DNS fałszywej informacji kojarzącej domenę z adresem IP.}
	\item{ARP -- rozsyłanie w sieci spreparowanych pakietów ARP zawierających fałszywe adresy MAC, co powoduje przesłanie danych w złe miejsce.}
\end{itemize}