\section{S4 -- Metody równoważenia obciążeń w systemach i sieciach komputerowych}

Równoważenie obciążenia w systemach nie jest banalnym tematem.
Wiele procesów wymaga dostępu do zasobów, w tym do czasu procesora.

Za kolejkowanie procesów odpowiada \textbf{planista.}
Jest to proces, którego zadaniem jest między innymi przydzielenie procesom miejsca w kolejce.

\subsection{Rodzaje planistów}
Wyróżniamy planistów
\begin{itemize}
	\item{krótkoterminowych -- odpowiadających za procesy oznaczone jako \textit{gotowe,}}
	\item{średnioterminowych -- odpowiadających za ładowanie i usuwanie procesów z pamięci (swap),}
	\item{długoterminowych -- odpowiadających za uruchamianie nowych procesóœ.}
\end{itemize}

\subsection{Algorytmy planowania}
Same algorytmy, wg. których działać może planista możemy podzielić na wywłaszczeniowe oraz niewywłaszczeniowe.

\subsubsection{Algorytmy wywłaszczeniowe}
To takie, które nie dopuszczają sytuacji, w której proces, który jest aktualnie obsługiwany, zostaje wywłaszczony -- przestaje się go obsługiwać przed zakończeniem.

Do takich algorytmów można zaliczyć kolejkowanie
\begin{itemize}
	\item{SJF - Shortest Job First,}
	\item{FIFO - First In First Out,}
	\item{LIFO - Last In First Out.}
\end{itemize}

\subsubsection{Algorytmy niewywłaszczeniowe}
Te algorytmy pozwalają, a nawet opierają się na tym, że proces zostaje wywłaszczony.
Można tutaj wspomnieć między innymi o
\begin{itemize}
	\item{Round-robin -- każdy proces dostaje kwant czasu, po którego upływie zostaje wywłaszczony,}
	\item{Shortest Remaining Time,}
	\item{Podejście priorytetowe -- obsługiwany jest proces o najwyższym priorytecie.}
\end{itemize}

