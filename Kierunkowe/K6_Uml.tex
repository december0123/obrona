\section{K6 - Język UML w projektowaniu oprogramowania}

UML w kontekście projektowania oprogramowania jest językiem pozwalającym na modelowanie i opisywanie systemów i ich części.

Pozwala na standaryzowany, graficzny zapis systemu z uwzględnieniem zarówno jego części konceptualnych, takich jak funkcje i procesy biznesowe, jak i obiektów fizycznych jakimi mogą być bazy danych czy warstwa sprzętowa.

W jaki sposób używa się UMLa do modelowania? Korzysta się z diagramów, które są podzielone ze względu na modelowanie strukturalne -- diagramy pakietów, klas, komponentów itd -- oraz behawioralne -- diagramy sekwencji, przypadków użycia, stanu itd.

UML jest przydatnym narzędziem nie tylko w procesie modelowania, ale również podczas szybkich rozmów między programistami kiedy chce się szybko nakreślić ideę.

\subsection{Diagram pakietów}
Pełni rolę organizacyjną. Zawiera w sobie inne elementy języka, zazwyczaj diagramy klas.

\subsection{Diagram klas}
Składa się z zamodelowanych klas oraz relacji między nimi.

Pojedynczą klasę umieszcza się w prostokącie podzielonym na trzy części
\begin{itemize}
	\item{nazwę,}
	\item{atrybuty,}
	\item{funkcje.}
\end{itemize}

Atrybuty oraz funkcje oznacza się modyfikatorem dostępu (np. public, private), składniki statyczne zaznacza się poprzez podkreślenie, a stereotypy -- elementy służące do doprecyzowania semantyki elementu, np. oznaczenie jako klucz główny -- umieszcza się w podwójnych nawiasach ostrych <<>>.

Relacje między klasami:
\begin{itemize}
	\item{\textbf{Zależność -- strzałka przerywana} -- podstawowa relacja mówiąca, że obiekt może w jakiś sposób korzystać lub wpływać na inne obiekty.}
	\item{\textbf{Asocjacja -- linia ciągła} -- obiekty wykorzystują inne obiekty przez dłuższy czas.}
	\item{\textbf{Agregacja częściowa -- pusta strzałka z rombem} -- }
	\item{\textbf{Dziedziczenie --} pusta strzałka -- określa hierarchię dziedziczenia.}
\end{itemize}
\subsection{Diagram przypadków użycia}
TODO

\subsection{Diagram sekwencji}
Pokazuje w sposób zgodny z intuicją kolejność wywołanych operacji i przepływ sterowania pomiędzy obiektami.

Diagram zbudowany jest z
\begin{itemize}
	\item{prostokątów, które oznaczają obiekty,}
	\item{pionowych linii życia tychże obiektów,}
	\item{komunikatów wymienianych między obiektami.}
\end{itemize}

Czas jest reprezentowany w postaci pionowej osi diagramu, a zajętość obiektu jest symbolizowana przez prostokąt umieszczony na jego linii życia.

Do komunikatów wymienianych między obiektami zalicza się między innymi
\begin{itemize}
	\item{wywołanie funkcji,}
	\item{powrót z wywołania,}
	\item{wywołanie asynchroniczne.}
\end{itemize}





