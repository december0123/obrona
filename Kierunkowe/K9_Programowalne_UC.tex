\sloppy\section{K9 -- Programowalne scalone układy cyfrowe PLD, CPLD oraz FPGA}

Programowalne układy scalone to układy scalone, których funkcjonalność jest definiowana przez użytkownika końcowego, a nie producenta. Pozwala to na zaprojektowanie, uruchomienie i testowanie urządzenia, w podobny sposób jak programowanie komputera.

Układy programowalne można podzielić ze względu na ich strukturę:
\begin{itemize}
\item PLD -- (programmable logic device) najprostsze,
\item CPLD -- complex PLD,
\item FPGA.
\end{itemize}

Układy PLD mogą realizować funkcje logiczne (tworząc układy kombinacyjne bądź sekwencyjne). Programowanie takich układów bazuje na ustawieniu odpowiednich bitów, by zostały zrealizowane dane funkcje. Każde urządzenie posiada dany zbiór wejść oraz wyjść do układu.

Najprostsze układy PLD pozwalają na tworzenie prostych układów. W ich skład wchodzą układy PLE, PAL, PLA. Ich budowa opiera się na matrycach funkcji AND oraz OR -- realizują funkcje postaci $(x_{1} \wedge x_{2} \wedge ...) \vee (x_{m} \wedge x_{m+1} \wedge ...) \vee ...$ lub $(x_{1} \vee x_{2} \vee ...) \wedge (x_{m} \vee x_{m+1} \vee ...) \wedge ...$ (dla sklerotyków: $\wedge$ -- AND, $\vee$ -- OR). Układy PAL posiadają programowalną matrycę bramek AND, bramki OR nie są programowalne. W układach PLE jest odwrotnie, natomiast układ PLA posiada obydwie matryce programowalne.

Układy CPLD są podobne do PLD, jednak są bardziej złożone. 

\textit{Dokończ mnie, nie zapominaj o mnie$\ldots$}
