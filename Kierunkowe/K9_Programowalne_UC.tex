\sloppy\section{K9 -- Programowalne scalone układy cyfrowe PLD, CPLD oraz FPGA}

Programowalne układy scalone to układy scalone, których funkcjonalność jest definiowana przez użytkownika końcowego, a nie producenta. Wszystkie wyprodukowane przez producenta układy są identyczne (co pozwala zmniejszyć koszty). Pozwala to na zaprojektowanie, uruchomienie i testowanie urządzenia. Programowanie odbywa się za pomocą tworzenia połączeń w istniejącej sieci ścieżek sygnałowych. 

Układy programowalne można podzielić ze względu na ich strukturę:
\begin{itemize}
\item PLD -- (programmable logic device) najprostsze,
\item CPLD -- complex PLD,
\item FPGA -- field-programmable gate array.
\end{itemize}

Układy PLD mogą realizować funkcje logiczne (tworząc układy kombinacyjne bądź sekwencyjne). Programowanie takich układów bazuje na ustawieniu odpowiednich bitów, by zostały zrealizowane dane funkcje. Każde urządzenie posiada dany zbiór wejść oraz wyjść do układu.

Najprostsze układy PLD (SPLD) pozwalają na tworzenie prostych układów. Do nich możemy zaliczyć układy PLE, PAL, PLA. Ich budowa opiera się na matrycach funkcji AND oraz OR -- realizują funkcje postaci $(x_{1} \wedge x_{2} \wedge ...) \vee (x_{m} \wedge x_{m+1} \wedge ...) \vee ...$ lub $(x_{1} \vee x_{2} \vee ...) \wedge (x_{m} \vee x_{m+1} \vee ...) \wedge ...$ (dla sklerotyków: $\wedge$ -- AND, $\vee$ -- OR). 

Układy PLA posiadają programowalne matryce AND oraz OR. Dowolna linia iloczynu logicznego z matrycy AND (tzw. \textit{term}) może zostać podłączony do realizacji dowolnej operacji sumy logicznej OR. Inna architektura, PAL posiada programowalną matrycę AND oraz stałą (nieprogramowalną) matrycę OR. Natomiast układy PLE posiadają stałą matrycę bramek AND oraz programowalną matrycę bramek OR.

Ze względu na ograniczone możliwości logiczne -- małą liczbę bramek oraz małą liczbę wejść/wyjść, układy SPLD mogły zastępować klasyczne obwody logiczne (czyli coś typu jak kleciliśmy na LUC-u).

Układy CPLD, jak nazwa wskazuje są bardziej złożone. Koncepcyjnie są podobne do SPLD, ale mają większe możliwości logiczne i funkcjonalne. Zbudowane są z zespołu struktur PAL połączonych ze sobą programowalną matrycą (switching matrix). Blok funkcjonalny CPLD posiada matrycę AND, makrokomórki oraz zadaną liczbę wyjść. Makrokomórki składają się zazwyczaj z bramek (np. AND, OR) oraz przerzutników. Sygnały sterujące makrokomórkami można podzielić na globalne (wspólne dla wszystkich makrokomórek), lokalne (wspólne dla zespołu połączonych ze sobą makrokomórek) oraz indywidualne (wpływają na działanie jednej makrokomórki). Pojedyncza makrokomórka realizuje prostą funkcję logiczną, większa ich liczba może być połączona ze sobą w bloki funkcyjne, tworząc funkcje z większą liczbą zmiennych. 

Najbardziej skomplikowane są układy FPGA. Mają one jeszcze większe możliwości logiczne w stosunku do CPLD oraz są szybsze w działaniu. Zbudowane są one z konfigurowalnych elementów logicznych (CLB). W skład elementu logicznego wchodzą zazwyczaj generator funkcji logicznych, przerzutnik i programowalne multipleksery. Generator funkcji logicznych określany jest jako LUT (look up table). Elementy logiczne są łączone w bloki. Połączenia między CLB są programowalne, podobnie jak w układach CPLD za pomocą programowalnej matrycy. LUT-y mają inne podejście do generowania funkcji -- zamiast programowania połączeń między bramkami, LUT-y działają bardziej jak tabele prawdy -- dla zadanego wejścia generują odpowiednie wyjście --- \textbf{to jest znacząca różnica między FPGA a CPLD}.

Do zalet struktur FPGA można zaliczyć elastyczność architektury (równoległość przetwarzania, dowolną szerokość ścieżki danych), wielokrotne użycie tych samych zasobów sprzętowych, czy też rekonfigurowalność.

Do programowania programowalnych układów scalonych wykorzystywane są języki opisu sprzętu, takie jak VHDL lub Verilog. Opis układu jest tworzony w sposób behawioralny. Narzędzie syntezy przetwarza taki zapis na konkretną realizację sprzętową. W porównaniu do układów ASIC (Full-Custom, Semi-Custom), takie układy są od nich szybsze. W przeciwieństwie do mikrokontrolerów, układy potrafią przetwarzać równolegle, a $\mu$C przetwarzają sekwencyjnie. Nie są one ograniczone zbiorem funkcji -- $\mu$C posiadają określony zestaw instrukcji. Wadą układów programowalnych w stosunku do nieprogramowalnych jest większy pobór mocy. Na układach programowalnych FPGA można zaprogramować procesor programowy -- PicoBlaze, MicroBlaze.

Układy PLD mają szerokie spektrum zastosowań. Można do tego zaliczyć logikę scalającą -- interfejs dla mikrokontrolerów umożliwiający współpracę z innymi modułami (pamięć, układy peryferyjne). Poprzez możliwość przetwarzania równoległego, układy PLD nadają się jako akceleratory sprzętowe. Akceleratory są wykorzystywane przy przetwarzaniu grafiki, dźwięku lub wideo.