\section{K1 - Paradygmat programowania obiektowego}

\textbf{Paradygmat} - zbiór mechanizmów działania, wzorów, które definiują sposób realizacji programu.

Programowanie obiektowe stanowi podejście do implementacji algorytmów, które opiera się na wykorzystaniu tak zwanych \textbf{obiektów}. Są to twory, które łączą w sobie \textit{dane} (pola obiektu) i \textit{zachowania} (metody obiektu) oraz komunikują się ze sobą w celu wykonania pewnych zadań.

Paradygmat ten ma stanowić ułatwienie w pisaniu i utrzymywaniu kodu, który może być używany wielokrotnie w różnych projektach jednak nie jest odpowiedzią na wszystkie problemy i oprócz swoich zalet ma również wady -- np. przez to, że obiekty posiadają wewnętrzny stan, programowanie współbieżne staje się o wiele trudniejsze, gdyż musimy zapobiegać wyścigom czy zagłodzeniu.

Pojęciami, na których się skupię będą:
\begin{itemize}
	\item Klasa
	\item Abstrakcja
	\item Enkapsulacja
	\item Dziedziczenie
	\item Polimorfizm
	\item Wzorce projektowe
\end{itemize}

\subsection{Klasa}
Stanowi pewien zbiór cech i zachowań danego obiektu, który jest jej instancją.

Przykładem może być klasa \textit{Pies,} która zawiera cechy takie jak oczy, pysk oraz umiejętność szczekania.

Języki programowania można podzielić ze względu na to, w jaki sposób pojęcie klasy jest zrealizowane. Na przykład w językach \textit{Java}, \textit{C\#} czy \textit{Python} każda klasa zawsze ma swoją klasę bazową (zazwyczaj dziedziczona niejawnie i jest to przeważnie klasa bazowa \textit{Object}). Język \textit{C++} na przykład, można nazwać językiem hybrydowym, gdyż umożliwia tworzenie obiektów, lecz nie zmusza do tego programisty (można programować zarówno strukturalnie jak i obiektowo). Istnieją także języki, których obiektowość opiera się na \textbf{prototypach} - nowe obiekty tworzone są w oparciu na istniejące już obiekty, nie zaś na podstawie zdefiniowanej klasy. Przykładem takiego języka jest większość języków interpretowanych, np. \textit{JavaScript}.

Głównym zyskiem obiektowości jest modularność - program można podzielić na mniejsze, gotowe do użycia moduły. Zwiększa to także czytelność kodu - łatwiej jest się odnaleźć w metodach przypisanych do obiektów niż w gąszczu funkcji.

\subsection{Abstrakcja}
Czyli poziom ogólności, pozwala nam na upraszczanie problemu poprzez zredukowanie właściwości do jedynie tych kluczowych dla algorytmu.

Dla przykładu - możemy mówić o \textit{Bazie danych} jako o fragmencie kodu, który będzie stanowił interfejs do komunikacji z pewną bazą danych, jednak nie ma to dla nas większego znaczenia w jaki sposób to będzie realizowane.

Przekładając to na banalny przykład z życia codziennego -- kierowca nie musi przechodzić szczegółowych szkoleń za każdym razem, gdy zmienia samochód.
Wystarczy mu jedynie podstawowa wiedza o jego działaniu, a takie rzeczy jak znajomość budowy silnika nie są mu potrzebne.

\subsection{Enkapsulacja}
Inaczej hermetyzacja, polega na celowym ukrywaniu wnętrza obiektów tak, aby zmiany w jego stanie mogły dokonać tylko metody wewnętrzne tego obiektu. Zwiększa to bezpieczeństwo kodu oraz odporność na błędy, a także pozwala podzielić kod na mniejsze fragmenty.

Problem pojawia się, gdy udostępnimy wersję klasy, która ma pewne pola publiczne.
Jeśli użytkownicy zaczną z tych pól korzystać, to aktualizacja może spowodować poważne problemy.

Enkapsulacja pomaga ustrzec nas przed niepożądanym korzystaniem z mechanizmów wewnątrz tworzonej przez nas klasy, a na użytek świata zewnętrznego pozwala nam wystawić tylko zdefiniowany przez nas interfejs.

Powracając do przykładu z samochodem -- kierowca powinien otwierać okno za pomocą guzika lub korbki, a nie poprzez wybijanie szyby.

Warto wspomnieć, że języki w różny sposób podchodzą do omawianych przeze mnie pojęć.
Python jest językiem, w którym pojęcie enkapsulacji praktycznie nie istnieje -- wszystko jest dostępne dla wszystkich i jeśli ktoś bardzo chce użyć zmiennych, które wyraźnie oznaczyliśmy jako do użytku wewnętrznego to może to zrobić na własną odpowiedzialność.

\subsection{Dziedziczenie}
Jest narzędziem, które dzięki odpowiednio zaprojektowanej relacji \textbf{klasa bazowa - klasa pochodna} pozwala na rozszerzanie funkcjonalności bez duplikowania kodu. Dzięki temu możemy tworzyć hierarchiczne struktury przechodząc od typów najbardziej ogólnych aż do tych szczegółowo opisujących dany obiekt bądź zjawisko.

W odniesieniu do samochodów -- możemy dostać wersję podstawową jakiegoś pojazdu, a następnie ją rozszerzyć o stylowe neony, naklejki z ogniem i spoilery.

\subsubsection{Dziedziczenie wielokrotne}
Tutaj znowu w zależności od języka możemy mieć możliwość skorzystania z mechanizmu dziedziczenia wielokrotnego lub nie. Językami to umożliwiającymi są na przykład \textit{C++} czy \textit{Python}. 

Polega ono na tym, że klasa pochodna może dziedziczyć po kilku klasach bazowych.
Jest to potężny mechanizm, jednak należy go używać z głową, gdyż może powodować problemy.

\subsection{Polimorfizm}
Pozwala nam na wyabstrahowanie pewnych zachowań od konkretnych typów danych. Dzięki niemu możemy wybrać zachowanie w zależności od kontekstu.

Dla odmiany rozpatrzmy przykład, w którym klasą \textbf{bazową} będzie \textbf{Zwierze,} a klasami \textbf{pochodnymi} będą \textbf{Pies} oraz \textbf{Ryba.}

Klasa bazowa ma zadeklarowaną metodę (funkcję, którą możemy wywołać na rzecz obiektu) o nazwie \textit{dajGlos.}

Klasy pochodne mogą tę funkcję zdefiniować po swojemu i w ten sposób po wywołaniu metody \textit{dajGlos} na obiekcie klasy \textit{Pies} usłyszymy szczekanie, a po wywołaniu metody na obiekcie klasy \textit{Ryba} usłyszymy tylko ciche bulgotanie, któremu towarzyszyć będzie bezgłośne osądzanie naszych wyborów życiowych.

Polimorfizm można podzielić na statyczny i dynamiczny. Polimorfizm \textbf{statyczny} (inaczej zwany \textit{wczesnym wiązaniem}) -- decyzja o użytym typie zostaje podjęta już na etapie kompilacji -- w języku \textit{C++} jest to realizowane za pomocą szablonów.  Polimorfizm \textbf{dynamiczny} (\textit{późne wiązanie}) -- wybór zostaje podjęty w czasie wykonywania programu -- w języku \textit{C++} zrealizowane przy użyciu wskaźników przeciążając metody wirtualne w klasach pochodnych.

\subsection{Wzorce projektowe}
Powstały, aby spisać często napotykane podczas programowania problemy oraz zdefiniować sprawdzone rozwiązania.

Jest to temat bardzo atrakcyjnych dla początkujących programistów i jest jednocześnie bardzo pomocny i bardzo niebezpieczny, ponieważ początkujący mogą chcieć korzystać ze wzorców gdzie tylko mogą, nie zważając na to czy faktycznie są one potrzebne.

Należy pamiętać o antywzorcach złotym młotku oraz srebrnym pocisku -- nie wszystko co się do tej pory sprawdziło gdzie indziej sprawdzi się w naszym przypadku, a to że doskonale znamy jakąś technologię nie znaczy, że jest ona zawsze odpowiednia.

Wzorce projektowe dzielimy wg trzech kategorii:
\begin{itemize}
	\item{konstrukcyjne -- opisujące proces tworzenia nowych obiektów, przykładowo \textbf{fabryka, singleton,}}
	\item{strukturalne -- opisujące struktury powiązanych ze sobą obiektów, przykładowo \textbf{adapter, dekorator,}}
	\item{behawioralne -- opisujące zachowanie i odpowiedzialność współpracujących ze sobą obiektów, przykładowo \textbf{strategia, null object.}}
\end{itemize}
