\sloppy\section{K8 -- Mechanizmy systemu operacyjnego wspomagające synchronizację procesów}

System operacyjny jest oprogramowaniem, którego jedną z głównych ról jest synchronizacja procesów.
Czym jest proces?
Jest to wykonujący się program, czyli zbiór instrukcji dla procesora przechowywany w postaci pliku.

Każdy proces wymaga przydziału zasobów komputera.
Czas procesora, pamięć, dostęp do urządzeń -- w mniejszym lub większym stopniu, proces zawsze wymaga pewnych zasobów.

Procesy mogą wykonywać się zupełnie niezależnie od siebie -- korzystając np. z innych urządzeń -- lub współbieżnie obok siebie korzystając z np. tych samych plików.

Jak łatwo się domyślić, procesy współbieżne wymagają synchronizacji ze strony zarządcy zasobów -- systemu operacyjnego.

Typowe problemy związane z procesami współbieżnymi to na przykład:
\begin{itemize}
\item problem sekcji krytycznej -- kiedy procesy współdzielą strukturę danych, a operacje na nich muszą być atomowe, tj. nie mogą zostać przerwane w środku wykonywania się,
\item problem czytelników i pisarzy -- synchronizacja dostępu do zasobów dla procesów dokonujących i niedokonujących w nich zmian,
\item problem producenta i konsumenta -- przetwarzanie danych przez proces może się odbyć jedynie, po otrzymaniu ich od innego procesu,
\item problem ucztujących filozofów -- procesy korzystają ze wspólnych zasobów, które pobierają i zwalniają wg potrzeb.
\end{itemize}

\textbf{Sekcja krytyczna} to sekwencja operacji wykonywanych na zasobie (np. pamięci, pliku), które muszą zostać wykonane w trybie wyłącznym przez jeden proces, tj. podczas ich wykonywania, inny proces nie może działać na zasobach przez niego zajętych.
Aby poprawnie rozwiązać problem sekcji krytycznej, należy zaimplementować algorytm, który będzie spełniał kilka warunków:
\begin{itemize}
\item instrukcje w sekcji krytycznej nie mogą być przeplatane -- może w niej być wyłącznie jeden proces,
\item nie można zakładać w jakiej kolejności i z jaką szybkością wykonają się dane procesy,
\item proces nie może zatrzymać się w sekcji krytycznej,
\item nie mogą występować zakleszczenia (jeden proces czeka na zwolnienie zasobów przez drugi, a drugi czeka na zwolnienie zasobów pierwszego, więc obydwa się zatrzymują),
\item każdy proces musi wejść do sekcji krytycznej (nie może wystąpić zagłodzenie).
\end{itemize}

Istnieją przeróżne mechanizmy synchronizacji implementowane przez jądro.

\textbf{Semafory} są typem danych, które służą do kontroli dostępu zasobów przez wiele procesów.
Semafory są zmienną całkowitą, która przyjmuje wartości nieujemne (tj. $\ge 0$) lub dla semaforów binarnych -- wartości logiczne.
Na semaforach można wykonywać dwie operacje -- zmniejszenie (zajęcie, podniesienie,) oraz zwiększenie (zwolnienie, opuszczenie).
Synchronizacja polega na blokowaniu procesu w operacji zajęcia semafora, gdy jego wartość po zajęciu jest ujemna, do czasu, aż wartość ta nie zostanie zwiększona do nieujemnej (np. przez inny proces, który zwolni semafor).
Wyróżnia się rodzaje semaforów takie jak:
\begin{itemize}
\item binarne -- zmienna przyjmuje tylko wartości \textit{true} (semafor otwarty) lub \textit{false} (semafor zamknięty),
\item zliczające -- zmienna przyjmuje wartości całkowite nieujemne, a aktualna wartość jest zwiększana/zmniejszana o 1 w wyniku zwolnienia/zajęcia semafora,
\item uogólnione -- semafor zliczający, ale może zwiększać/zmniejszać wartość o dowolną liczbę (oczywiście, wartość wciąż musi być nieujemna).
\end{itemize}

\textbf{Muteksy} (\texttt{\textbf{mut}ual \textbf{ex}clusion} -- wzajemne wykluczanie) są szczególnym przypadkiem semaforów. Muteks obsługują operacje blokowania i zwalniania, jednak mogą być zwolnione tylko przez proces, które je zajął.

\textbf{Spinlocki} -- podobne do semaforów, jednak oczekiwanie na zwolnienie blokady odbywa się na zasadzie aktywnego czekania, przez co zajmują czas procesora, może wystąpić zagłodzenie lub czekać w nieskończoność.

\textbf{Monitory} są strukturalnym narzędziem synchronizacji wątków. Składają się ze zmiennych oraz procedur operujących na nich, zebrane w jeden moduł. Dostęp do zmiennych jest możliwy wyłącznie za pomocą procedur monitora, a w danej chwili tylko jeden proces może wywoływać procedury monitora. Gdy inny proces wywoła procedurę monitora, to będzie on zablokowany do chwili opuszczenia monitora przez pierwszy proces. Istnieje możliwość wstrzymania i wznowienia procedur monitora za pomocą zmiennych warunkowych. Na zmiennych warunkowych można wykonywać operacje \texttt{wait} (wstrzymanie procesu i umieszczenie go na końcu kolejki) oraz \texttt{signal} (odblokowanie jednego z oczekujących procesów). Procesy oczekujące na wejście do monitora zorganizowane są w kolejkę FIFO.