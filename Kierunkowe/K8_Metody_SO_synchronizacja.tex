\sloppy\section{K8 -- Mechanizmy systemu operacyjnego wspomagające synchronizację procesów}

Zarządzanie procesami to ważne zadanie systemu operacyjnego. \textbf{Program} to statyczny obiekt, zbiór instrukcji dla procesora, przechowywany w postaci pliku. \textbf{Proces} natomiast to wykonujący się program. Każdy proces wymaga przydziału zasobów (czas procesora, pamięć, pliki) do wykonania swojego zadania. Obsługa procesów jest wykonywana przez jądro systemu operacyjnego.

Problem synchronizacji procesów pojawia się przy procesach współbieżnych, które ze sobą współpracują. Przykłady, w których wymagana jest synchronizacja procesów to:
\begin{itemize}
\item problem sekcji krytycznej -- procesy współdzielą strukturę danych, a operacje na nich muszą być atomowe,
\item problem czytelników i pisarzy -- synchronizacja dostępu do zasobów dla procesów dokonujących i niedokonujących w nich zmian,
\item problem producenta i konsumenta -- wynik działania jednego procesu wpływa na działanie drugiego procesu -- przetwarzanie danych przez drugi proces może się odbyć jedynie, gdy zostanie dane zostaną przekazane przez pierwszy proces,
\item problem ucztujących filozofów -- procesy korzystają ze wspólnych zasobów, które pobierają i zwalniają wg potrzeb.
\end{itemize}

\textbf{Sekcja krytyczna} to sekwencja operacji wykonywanych na zasobie (np. pamięci, pliku), który musi być wykonywany w trybie wyłącznym przez jeden proces. Poprawnym rozwiązaniem sekcji krytycznej jest:
\begin{itemize}
\item instrukcje w sekcji krytycznej nie mogą być przeplatane -- może w niej być wyłącznie jeden proces,
\item nie można zakładać w jakiej kolejności i z jaką szybkością wykonają się dane procesy,
\item proces nie może zatrzymać się w sekcji krytycznej,
\item nie mogą występować zakleszczenia (jeden proces czeka na zwolnienie zasobów przez drugi, a drugi czeka na zwolnienie zasobów pierwszego, więc obydwa się zatrzymują),
\item każdy proces musi wejść do sekcji krytycznej (nie może wystąpić zagłodzenie).
\end{itemize}

Istnieją przeróżne mechanizmy synchronizacji implementowane przez jądro.

\textbf{Semafory} są typem danych, które służą do kontroli dostępu zasobów przez wiele procesów. Semafory są zmienną całkowitą, która przyjmuje wartości nieujemne (tj. $\ge 0$) lub dla semaforów binarnych -- wartości logiczne. Semafor ma wartość początkową nieujemną. Na semaforach można wykonywać dwie operacje -- zmniejszenie (zajęcie, podniesienie,) oraz zwiększenie (zwolnienie, opuszczenie). Synchronizacja polega na blokowaniu procesu w operacji zajęcia semafora do czasu, aż wartość nie zostanie zwiększona (np. przez inny proces, który zwolni semafor). Wyróżnia się rodzaje semaforów takie jak:
\begin{itemize}
\item binarne -- zmienna przyjmuje tylko wartości \textit{true} (semafor otwarty) lub \textit{false} (semafor zamknięty),
\item zliczające -- zmienna przyjmuje wartości całkowite nieujemne, a aktualna wartość jest zwiększana/zmniejszana o 1 w wyniku zwolnienia/zajęcia semafora,
\item uogólniony -- semafor zliczający, ale może zwiększać/zmniejszać wartość o dowolną liczbę (oczywiście, aby wartość wciąż była nieujemna).
\end{itemize}

\textbf{Muteksy} (\texttt{\textbf{mut}ual \textbf{ex}clusion} -- wzajemne wykluczanie) są szczególnym przypadkiem semaforów. Muteks może być zablokowany lub odblokowany. Muteksy mogą być zwalniane tylko i wyłącznie przez proces, który zajął dany muteks. Dzięki temu wyeliminowane jest zjawisko zakleszczenia oraz niespójności danych.

\textbf{Spinlocki} -- podobne do semaforów, jednak oczekiwanie na zwolnienie blokady odbywa się na zasadzie aktywnego czekania, przez co zajmują czas procesora, może wystąpić zagłodzenie lub czekać w nieskończoność.

\textbf{Monitory} są strukturalnym narzędziem synchronizacji wątków. Składają się ze zmiennych oraz procedur operujących na nich, zebrane w jeden moduł. Dostęp do zmiennych jest możliwy wyłącznie za pomocą procedur monitora, a w danej chwili tylko jeden proces może wywoływać procedury monitora. Gdy inny proces wywoła procedurę monitora, to będzie on zablokowany do chwili opuszczenia monitora przez pierwszy proces. Istnieje możliwość wstrzymania i wznowienia procedur monitora za pomocą zmiennych warunkowych. Na zmiennych warunkowych można wykonywać operacje \texttt{wait} (wstrzymanie procesu i umieszczenie go na końcu kolejki) oraz \texttt{signal} (odblokowanie jednego z oczekujących procesów). Procesy oczekujące na wejście do monitora zorganizowane są w kolejkę FIFO.