\section{K5 - Ocena złożoności algorytmów}

\textbf{Złożoność algorytmu} jest miarą ilości zasobów potrzebnych do rozwiązania danego problemu o określonym rozmiarze. Dla przykładu w algorytmie rozkładu liczb na czynniki pierwsze zauważyć można, że im większa liczba, tym dłużej będziemy ją rozbijać i wykonamy więcej obliczeń. Faktem zatem jest, że im większy rozmiar danych wejściowych, tym więcej zasobów potrzebujemy do rozwiązania problemu. Złożoność algorytmu można więc określić mianem \textbf{funkcji rozmiaru danych wejściowych}.

Wyróżnić możemy kilka \textbf{klas złożoności}, czyli grup zagadnień o podobnej złożoności obliczeniowej. Są to:
\begin{enumerate}
	\item \textbf{NP} (\textit{nondeterministic polynomial}) - problemy decyzyjne rozwiązywalne niedeterministycznym algorytmem 
wielomianowym.
	\item \textbf{P} (\textit{deterministic polynomial}) - problemy decyzyjne, które można rozwiązać deterministycznym algorytmem 
o złożoności wielomianowej.
	\item \textbf{NP-zupełny} - problemy decyzyjne, których znalezienie rozwiązania nie jest możliwe w czasie wielomianowym*.
	\item \textbf{NP-trudny} (silnie NP-zupełny) - samo sprawdzenie rozwiązania problemu jest co najmniej tak trudne jak każdego innego problemu NP.
\end{enumerate}

