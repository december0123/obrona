\section{K3 - Normalizacja schematu bazy danych}

Normalizacja schematu bazy danych, czyli sprowadzanie schematu do jednej z postaci normalnych jest dokonywana, aby przeciwdziałać lub też zapobiegać problemom, które pojawiają się podczas cyklu życia bazy danych.

Przykładowe problemy związane z użytkowaniem i utrzymaniem bazy danych:
\begin{itemize}
	\item{niespójność danych,}
	\item{zbyt skomplikowane zapytania,}
	\item{anomalie spowodowane aktualizacją danych.}
\end{itemize}

Normalizacja najczęściej sprowadza się do rozbijania tabel na mniejsze, jednak co ważne, nie wpływa ona na dane, a jedynie na sposób w jaki je przechowujemy.

Sprowadzenie schematu do którejś z postaci normalnych polega na sprawieniu, by schemat spełniał warunki określone przez postać do której dążymy oraz przez wszystkie poprzednie. Przykładowo: schemat jest w 3NF jeśli spełnia warunki 3NF, 2NF oraz 1NF.

Pan Codd, który był twórcą tego pojęcia, wymyślił trzy postaci normalne:
\begin{itemize}
	\item{1NF} - Przeciwdziała redundancji
\end{itemize}