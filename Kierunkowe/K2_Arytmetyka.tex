\sloppy\section{K2 -- Arytmetyka stało- i zmiennoprzecinkowa}

Każda informacja w komputerze może zostać przedstawiona za pomocą podstawowych jednostek -- bitów. W przypadku liczb, można je reprezentować za pomocą typów \{stało-,zmienno-\}przecinkowych. Różnią się cechami takimi jak:
\begin{itemize}
\item sposób zapisu,
\item zakres wartości,
\item precyzja operacji.
\end{itemize}

Typy stałoprzecinkowe posiadają wiele reprezentacji. 

\subsection{Naturalny kod binarny (NBC)}
Jest to system pozycyjny o podstawie 2. Liczby zapisane w tej reprezentacji nie posiadają znaku, nie można za jego pomocą zapisać ułamków. Wartość liczby można zaprezentować za pomocą wzoru (liczba \textit{n-bitowa}):
\begin{equation}
x = \sum_{i=0}^{n-1}2^{i} \times x_{i}
\end{equation}
Na przykład:
\begin{table}[H]
\centering
\begin{tabular}{|c|c|} \hline
DEC	&	BIN (NKB)	\\ \hline
0	&	00	\\ \hline
1	&	01	\\ \hline
2	&	10	\\ \hline
3	&	11	\\ \hline
\end{tabular}
\end{table}

\subsection{Kod uzupełnieniowy U2}
W celu reprezentacji liczby całkowitej potrzebne jest zdefiniowanie znaku liczby. Wartość liczb zapisuje się podobnie jak w wypadku NKB, jednak znak liczby jest określony przez bit najbardziej znaczący (0 -- liczba dodatnia, 1 -- liczba ujemna) (liczba \textit{n-bitowa}):
\begin{equation}
x = (-1)^{n-1}+\sum_{i=0}^{n-2}2^{i} \times x_{i}
\end{equation}

W dwójkowym systemie uzupełnieniowym porządek kodów arytmetycznych jest zachowany w całym zakresie. $\{0, x_{n-2},\ldots,x_{1},x_{0}\}$ reprezentuje dodatnią liczbę $x$, a $\{1, x_{n-2},\ldots,x_{1},x_{0}\}$ to zapis liczby \textbf{większej o $x$ od najmniejszej liczby $-2^{n-1}$}.

Podobnie jak dla NKB, nie jest możliwe zastosowanie tutaj zapisu ułamków.

\subsection{Typ stałoprzecinkowy}
Reprezentacja ta jest podobna do kodu uzupełnieniowego, jednakże można tutaj określić pozycję przecinka. Dla liczb o podstawie $\beta$ oraz ustalonej liczbie pozycji części ułamkowej -- $r$, wartość każdej liczby jest podana z dokładnością $\beta^{-r}$. Można ją przedstawić za pomocą iloczynu liczby całkowitej, wtedy zapis wygląda następująco:
\begin{equation}
\{x_{m},x_{m-1},\ldots,x_{1},x_{0},\ldots,x_{-r}\}
\end{equation}

Podczas operacji z liczbami stałoprzecinkowymi można otrzymać przeniesienie w wypadku operacji dodawania lub mnożenia.

\subsection{Inne systemy stałoprzecinkowe}
\begin{itemize}
\item system obciążony,
\begin{itemize}
\item[+] unikatowa reprezentacja zera,
\item[+] zgodność uporządkowania liczb i ich reprezentacji,
\item[-] wymagana korekcja wyników działań arytmetycznych,
\item[-] problematyczne przy dzieleniu i mnożeniu.
\end{itemize}
\item system ze znakowaną cyfrą.
\end{itemize}

\subsection{Typ zmiennoprzecinkowy}

Liczba zmiennoprzecinkowa jest reprezentowana przez 4 liczby ($S$, $\beta$, $M$, $E$):
\begin{itemize}
\item $S$ -- znak,
\item $\beta$ -- podstawa,
\item $E$ -- wykładnik,
\item $M$ -- mantysa.
\end{itemize}
\begin{equation}
x = (-1)^{S} \times \beta^{E} \times M
\end{equation}

W przypadku liczb zmiennoprzecinkowych ważna jest kolejność działań!

Liczby zmiennoprzecinkowe zawierają zarezerwowane wartości dla liczb specjalnych.

Standard IEEE754 wymaga, aby liczby były znormalizowane (oprócz wartości specjalnych). Standard wymaga również, aby mantysa była w zakresie $1 \le |M| < 2$. Dzięki temu można ukryć najwyższy bit (mantysa jest postaci $1.bbb\ldots bb$).

Jeśli liczba jest ujemna, to jej pierwszy bit (znaku) przyjmuje wartość 1, w wypadku dodatniej 0. Kolejne bity przypadają na wykładnik w kodzie +N ($N = 2^{k-1}-1$), kolejnymi bitami jest mantysa. Nigdy nie będzie ona równa 0. Przyjęto, że zero oprócz bitu znaku zawiera same zera (tak więc mamy tu zero dodatnie i ujemne). $\pm\infty$ -- wszystkie bity wykładnika wynoszą 1, mantysa 0. Dla nie-liczb wykładnik również przyjmie same jedynki, natomiast mantysę $\ne$ 0.

Liczby zdenormalizowane posiadają w wykładniku same zera, natomiast mantysa jest postaci $0.bbb\ldots bb$.

\begin{table}[H]
\centering
\caption{Liczba zmiennoprzecinkowa pojedynczej precyzji (32 bity)}
\footnotesize
\addtolength{\tabcolsep}{-4.75pt}
\begin{tabular}{ccccccccccccccccccccccccccccccccc}
\rotatebox[origin=c]{90}{Bit}                     &                        &                        &                        &                        &                        &                        &                         &                        &                        &                        &                        &                        &                        &                        &                         &                        &                        &                        & & & & & & & & & & & & & &           \\
\multicolumn{1}{|l}{31} &                        &                        &                        &                        &                        &                        & \multicolumn{1}{l|}{24} & 23                     &                        &                        &                        &                        &                        &                        & \multicolumn{1}{l|}{16} & 15                     &                        &                        & & & & & \multicolumn{1}{l|}{8} & 7 & & & & & & & & \multicolumn{1}{c|}{0}              \\ \hline
\multicolumn{1}{|c|}{\cellcolor{red!75}S} & \multicolumn{1}{c|}{\cellcolor{yellow!75}E} & \multicolumn{1}{c|}{\cellcolor{yellow!75}E} & \multicolumn{1}{c|}{\cellcolor{yellow!75}E} & \multicolumn{1}{c|}{\cellcolor{yellow!75}E} & \multicolumn{1}{c|}{\cellcolor{yellow!75}E} & \multicolumn{1}{c|}{\cellcolor{yellow!75}E} & \multicolumn{1}{c|}{\cellcolor{yellow!75}E}  & \multicolumn{1}{c|}{\cellcolor{yellow!75}E} & \multicolumn{1}{c|}{\cellcolor{green!75}M} & \multicolumn{1}{c|}{\cellcolor{green!75}M} & \multicolumn{1}{c|}{\cellcolor{green!75}M} & \multicolumn{1}{c|}{\cellcolor{green!75}M} & \multicolumn{1}{c|}{\cellcolor{green!75}M} & \multicolumn{1}{c|}{\cellcolor{green!75}M} & \multicolumn{1}{c|}{\cellcolor{green!75}M}  & \multicolumn{1}{c|}{\cellcolor{green!75}M} & \multicolumn{1}{c|}{\cellcolor{green!75}M} & \multicolumn{1}{c|}{\cellcolor{green!75}M} & \multicolumn{1}{c|}{\cellcolor{green!75}M} & \multicolumn{1}{c|}{\cellcolor{green!75}M} & \multicolumn{1}{c|}{\cellcolor{green!75}M} & \multicolumn{1}{c|}{\cellcolor{green!75}M} & \multicolumn{1}{c|}{\cellcolor{green!75}M} & \multicolumn{1}{c|}{\cellcolor{green!75}M} & \multicolumn{1}{c|}{\cellcolor{green!75}M} & \multicolumn{1}{c|}{\cellcolor{green!75}M} & \multicolumn{1}{c|}{\cellcolor{green!75}M} & \multicolumn{1}{c|}{\cellcolor{green!75}M} & \multicolumn{1}{c|}{\cellcolor{green!75}M} & \multicolumn{1}{c|}{\cellcolor{green!75}M} & \multicolumn{1}{c|}{\cellcolor{green!75}M} & \multicolumn{1}{c|}{\cellcolor{green!75}M} \\ \hline
\rotatebox[origin=c]{90}{Znak~}                        & \rotatebox[origin=c]{90}{Wykładnik~}                       &                        &                        &                        &                        &                        &                         &                        & \rotatebox[origin=c]{90}{Mantysa~}                       &                        &                        &                        &                        &                        &                         &                        &                        &                        &                       
\end{tabular}
\normalsize
\end{table}

\begin{table}[H]
\centering
\caption{Liczby w standardzie IEEE754}
\begin{tabular}{|c|c|c|c|}
\hline
\textbf{Znak} & \textbf{Wykładnik} & \textbf{Mantysa} & \textbf{Wartość} \\ \hline
0 & $00\cdots00$ & $00\cdots00$ & $+0$ \\ \hline
0 & $00\cdots00$ & \begin{tabular}[c]{@{}c@{}}$00\cdots01$\\$\vdots$\\$11\cdots11$\end{tabular} & \begin{tabular}[c]{@{}c@{}}Dodatnia zdenormalizowana\\($0.f\times 2^{-N+1}$)\end{tabular} \\ \hline
0 & \begin{tabular}[c]{@{}c@{}}$00\cdots01$\\$\vdots$\\$11\cdots10$\end{tabular} & XX$\cdots$XX & \begin{tabular}[c]{@{}c@{}}Dodatnia znormalizowana\\($1.f\times 2^{E-N}$)\end{tabular} \\ \hline
0 & $11\cdots11$ & $00\cdots00$ & $+\infty$ \\ \hline
0 & $11\cdots11$ & \begin{tabular}[c]{@{}c@{}}$00\cdots01$\\$\vdots$\\$01\cdots11$\end{tabular} & SNaN \\ \hline
0 & $11\cdots11$ & 1X$\cdots$XX & QNaN \\ \hline
1 & $00\cdots00$ & $00\cdots00$ & $-0$ \\ \hline
1 & $00\cdots00$ & \begin{tabular}[c]{@{}c@{}}$00\cdots01$\\$\vdots$\\$11\cdots11$\end{tabular} & \begin{tabular}[c]{@{}c@{}}Ujemna zdenormalizowana\\($-0.f\times 2^{-N+1}$)\end{tabular} \\ \hline
1 & \begin{tabular}[c]{@{}c@{}}$00\cdots01$\\$\vdots$\\$11\cdots10$\end{tabular} & XX$\cdots$XX & \begin{tabular}[c]{@{}c@{}}Ujemna znormalizowana\\($-1.f\times 2^{E-N}$)\end{tabular} \\ \hline
1 & $11\cdots11$ & $00\cdots00$ & $-\infty$ \\ \hline
1 & $11\cdots11$ & \begin{tabular}[c]{@{}c@{}}$00\cdots01$\\$\vdots$\\$01\cdots11$\end{tabular} & SNaN \\ \hline
1 & $11\cdots11$ & 1X$\cdots$XX & QNaN \\ \hline
\end{tabular}
\end{table}
N -- obciążenie wykładnika, X -- dowolna wartość.

\begin{table}[H]
\centering
\caption{Specjalne operacje arytmetyczne}
\begin{tabular}{|c|c|}
\hline
\textbf{Operacja} & \textbf{Wynik} \\ \hline
$\frac{n}{\pm\infty}$ & 0  \\ \hline
$\pm\infty\times\pm\infty$ & $\pm\infty$  \\ \hline
$\frac{\pm X}{0}$ & $\pm\infty$  \\ \hline
$X\times\pm\infty$ & $\pm\infty$  \\ \hline
\begin{tabular}[c]{@{}c@{}}$\infty+\infty$\\$\infty- -\infty$\end{tabular} & $+\infty$  \\ \hline
\begin{tabular}[c]{@{}c@{}}$-\infty-\infty$\\$-\infty+ -\infty$\end{tabular} & $-\infty$  \\ \hline
\begin{tabular}[c]{@{}c@{}}$\infty-\infty$\\$-\infty+ \infty$\end{tabular} & NaN \\ \hline
$\frac{\pm 0}{\pm 0}$ & NaN \\ \hline
$\frac{\pm\infty}{\pm\infty}$ & NaN \\ \hline
$\infty\pm 0$ & NaN \\ \hline
\end{tabular}
\end{table}

Standard wyróżnia następujące precyzje:
\begin{itemize}
\item połowicza (16 bit),
\item pojedyncza (32 bit),
\item podwójna (64 bit),
\item podwójna rozszerzona (80 bit).
\end{itemize}

Standard również wyróżnia wyjątki:
\begin{itemize}
\item invalid operation --- niewłaściwa operacja,
\item division by zero --- dzielenie przez zero,
\item overflow --- nadmiar -- liczba jest za duża,
\item underflow --- niedomiar -- utrata precyzji przy liczbach bliskich zero,
\item inexact --- niedokładność -- podczas operacji zaokrąglania.
\end{itemize}

\subsection{Podsumowanie}
\begin{table}[H]
\centering
\begin{tabular}{|l|l|l|}
\hline
\multicolumn{1}{|c|}{\textbf{Cecha}}                          & \multicolumn{1}{c|}{\textbf{L. stałoprzecinkowe}}                                               & \multicolumn{1}{c|}{\textbf{L.zmiennoprzecinkowe}}                                                             \\ \hline
Reprezentacja                                                 & \begin{tabular}[c]{@{}l@{}}Tak jak w systemie \\ pozycyjnym lub \\ uzupełnieniowym\end{tabular} & \begin{tabular}[c]{@{}l@{}}Reprezentacja przez:\\ znak, wykładnik, mantysę\end{tabular}                        \\ \hline
Dokładność                                                    & \begin{tabular}[c]{@{}l@{}}Mały zakres, \\ dokładne wyniki\end{tabular}                         & \begin{tabular}[c]{@{}l@{}}Duży zakres, możliwe \\ zaokrąglenia, utrata precyzji\end{tabular}                  \\ \hline
\begin{tabular}[c]{@{}l@{}}Obsługa \\ błędów\end{tabular}     & \begin{tabular}[c]{@{}l@{}}Może wystąpić \\ przepełnienie\end{tabular}                          & \begin{tabular}[c]{@{}l@{}}Zaokrąglenia, błędne operacje, \\ przepełnienie, niedomiar\end{tabular}             \\ \hline
\begin{tabular}[c]{@{}l@{}}Wartości \\ specjalne\end{tabular} & \multicolumn{1}{c|}{---}                                                                        & \begin{tabular}[c]{@{}l@{}}+0, -0, $+\infty$, $-\infty$,\\ QNaN, SNaN, liczby \\ zdenormalizowane\end{tabular} \\ \hline
Arytmetyka                                                    & \begin{tabular}[c]{@{}l@{}}Wszystkie twierdzenia \\ prawdziwe\end{tabular}                      & \begin{tabular}[c]{@{}l@{}}Kolejność działań \\ wpływa na wynik\end{tabular}                                   \\ \hline
\end{tabular}
\end{table}