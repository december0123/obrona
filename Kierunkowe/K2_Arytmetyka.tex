\sloppy\section{K2 -- Arytmetyka stało- i zmiennoprzecinkowa}

Ludzie spodziewają się, że komputery będą dokładne i nieomylne.
Spodziewają się, że po wpisaniu 0.1 + 0.2 kalkulator odpowie: 0.3.

Prędzej czy później każdy programista próbuje czegoś takiego i natrafia na problem: komputer twierdzi, że wynik tego dodawania to np. 0.30000001.
Skąd takie rzeczy?


Każda informacja w komputerze jest przedstawiona za pomocą podstawowych jednostek -- bitów. W przypadku liczb, można je reprezentować za pomocą typów \{stało-,zmienno-\}przecinkowych. Różnią się cechami takimi jak:
\begin{itemize}
\item sposób zapisu,
\item zakres wartości,
\item precyzja operacji.
\end{itemize}

Typy stałoprzecinkowe posiadają wiele reprezentacji. 

\subsection{Naturalny kod binarny (NBC)}
Jest to system pozycyjny o podstawie 2. Liczby zapisane w tej reprezentacji nie posiadają znaku, nie można za jego pomocą zapisać ułamków. Wartość liczby można zaprezentować za pomocą wzoru (liczba \textit{n-bitowa}):
\begin{equation}
x = \sum_{i=0}^{n-1}2^{i} \times x_{i}
\end{equation}
Na przykład:
\begin{table}[H]
\centering
\begin{tabular}{|c|c|} \hline
DEC	&	BIN (NKB)	\\ \hline
0	&	00	\\ \hline
1	&	01	\\ \hline
2	&	10	\\ \hline
3	&	11	\\ \hline
\end{tabular}
\end{table}

\subsection{Kod uzupełnieniowy U2}
W celu reprezentacji liczby całkowitej potrzebne jest zdefiniowanie znaku liczby. Wartość liczb zapisuje się podobnie jak w wypadku NKB, jednak znak liczby jest określony przez bit najbardziej znaczący (0 -- liczba dodatnia, 1 -- liczba ujemna).
Aby otrzymać wartość n-bitowej liczby zapisanej w kodzie U2 można skorzystać z następującego wzoru:
\begin{equation}
x = (-1)^{n-1}+\sum_{i=0}^{n-2}2^{i} \times x_{i}
\end{equation}

Jak wynika ze wzoru, o ile liczba dodatnia wygląda tak samo jak w NKB, to postać liczby ujemnej zapisanej w U2 wcale nie jest taka sama jak w NKB z dodaną jedynką na przodzie.

\begin{table}[H]
\centering
\begin{tabular}{|c|c|c|} \hline
DEC	&	NKB		&	U2	\\ \hline
-3	&	---		&	101\\ \hline
-2	&	---		&	110\\ \hline
-1	&	---		&	111\\ \hline
0	&	000		&	000\\ \hline
1	&	001		&	001\\ \hline
2	&	010		&	010\\ \hline
3	&	011		&	011\\ \hline
\end{tabular}
\end{table}

Dwa łatwe sposoby na odczytanie wartości liczby ujemnej w U2:
\begin{itemize}
	\item{Neguj wszystkie bity i dodaj 1.}
	\item{Potraktuj najstarszy bit jak ujemną wartość na tej pozycji i dodaj do niej pozostałe bity.}
\end{itemize}

W dwójkowym systemie uzupełnieniowym porządek kodów arytmetycznych jest zachowany w całym zakresie. $\{0, x_{n-2},\ldots,x_{1},x_{0}\}$ reprezentuje dodatnią liczbę $x$, a $\{1, x_{n-2},\ldots,x_{1},x_{0}\}$ to zapis liczby \textbf{większej o $x$ od najmniejszej liczby $-2^{n-1}$}.

Podobnie jak dla NKB, nie jest możliwe zastosowanie tutaj zapisu ułamków.

\subsection{Typ stałoprzecinkowy}
Reprezentacja ta jest podobna do kodu uzupełnieniowego, jednakże można tutaj określić pozycję przecinka. Dla liczb o podstawie $\beta$ oraz ustalonej liczbie pozycji części ułamkowej -- $r$, wartość każdej liczby jest podana z dokładnością $\beta^{-r}$. Można ją przedstawić za pomocą iloczynu liczby całkowitej, wtedy zapis wygląda następująco:
\begin{equation}
\{x_{m},x_{m-1},\ldots,x_{1},x_{0},\ldots,x_{-r}\}
\end{equation}

Podczas operacji z liczbami stałoprzecinkowymi można otrzymać przeniesienie w wypadku operacji dodawania lub mnożenia.

\subsection{Inne systemy stałoprzecinkowe}
\begin{itemize}
\item system obciążony,
\begin{itemize}
\item[+] unikatowa reprezentacja zera,
\item[+] zgodność uporządkowania liczb i ich reprezentacji,
\item[-] wymagana korekcja wyników działań arytmetycznych,
\item[-] problematyczne przy dzieleniu i mnożeniu.
\end{itemize}
\item system ze znakowaną cyfrą.
\end{itemize}

\subsection{Typ zmiennoprzecinkowy}
\subsubsection{Budowa}
Przedstawione do tej pory systemy były \textbf{stało}przecinkowe.
Znaczy to tyle, że miały z góry określoną precyzję i rozmiar.

Jedną z wad typów stałoprzecinkowych jest ilość miejsca potrzebna do przechowywania liczb: aby przechować liczbę z zakresu <0; 4 294 967 295> potrzebujemy aż 32 bitów pamięci.

Wyobraźmy sobie, że potrzebne nam są liczby rzędu rozmiaru wszechświata albo wielkości atomu.

Gdybyśmy takie liczby chcieli zapisać w typie stałoprzecinkowym to musielibyśmy zmarnować ogromne ilości pamięci na przechowywanie zer.

Rozwiązanie? Notacja naukowa i zmienny przecinek.

Liczba zmiennoprzecinkowa jest reprezentowana przez 4 liczby ($S$, $\beta$, $M$, $E$):
\begin{itemize}
\item $S$ -- znak,
\item $\beta$ -- podstawa,
\item $E$ -- wykładnik,
\item $M$ -- mantysa.
\end{itemize}
\begin{equation}
x = (-1)^{S} \times \beta^{E} \times M
\end{equation}

Łatwo zauważyć, że jest to po prostu notacja naukowa, lecz jak tak właściwie taka liczba wygląda?

Pierwszy bit jest bitem znaku -- 0 dla liczb dodatnich i 1 dla liczb ujemnych. Tak jest zawsze.

Kolejne bity to bity wykładnika i to ile tych bitów jest zależy od typu (precyzji).
Dla liczby pojedynczej precyzji, czyli typu 32-bitowego, liczba bitów wykładnika wynosi 8.

\textbf{Ważne:} wykładnik jest zapisany w systemie obciążonym +N ($N = 2^{k-1}-1$, gdzie k to liczba bitów wykładnika) -- dla 32-bitowej liczby jest to system +127.

Pozostałe bity to liczby części ułamkowej.
Ważne jest tutaj pojęcie liczby znormalizowanej, czyli takiej, dla której przyjmujemy, że część ułamkowa ma na początku ukrytą jedynkę -- jest więc postaci 1.bb...bb, a nie 0.bb...bb.
Więcej o tym za chwilę.
\textbf{Ważne:} mantysa zawsze jest liczbą dodatnią.


\begin{table}[H]
\centering
\caption{Liczba zmiennoprzecinkowa pojedynczej precyzji (32 bity)}
\footnotesize
\addtolength{\tabcolsep}{-4.75pt}
\begin{tabular}{ccccccccccccccccccccccccccccccccc}
\rotatebox[origin=c]{90}{Bit}                     &                        &                        &                        &                        &                        &                        &                         &                        &                        &                        &                        &                        &                        &                        &                         &                        &                        &                        & & & & & & & & & & & & & &           \\
\multicolumn{1}{|l}{31} &                        &                        &                        &                        &                        &                        & \multicolumn{1}{l|}{24} & 23                     &                        &                        &                        &                        &                        &                        & \multicolumn{1}{l|}{16} & 15                     &                        &                        & & & & & \multicolumn{1}{l|}{8} & 7 & & & & & & & & \multicolumn{1}{c|}{0}              \\ \hline
\multicolumn{1}{|c|}{\cellcolor{red!75}S} & \multicolumn{1}{c|}{\cellcolor{yellow!75}E} & \multicolumn{1}{c|}{\cellcolor{yellow!75}E} & \multicolumn{1}{c|}{\cellcolor{yellow!75}E} & \multicolumn{1}{c|}{\cellcolor{yellow!75}E} & \multicolumn{1}{c|}{\cellcolor{yellow!75}E} & \multicolumn{1}{c|}{\cellcolor{yellow!75}E} & \multicolumn{1}{c|}{\cellcolor{yellow!75}E}  & \multicolumn{1}{c|}{\cellcolor{yellow!75}E} & \multicolumn{1}{c|}{\cellcolor{green!75}M} & \multicolumn{1}{c|}{\cellcolor{green!75}M} & \multicolumn{1}{c|}{\cellcolor{green!75}M} & \multicolumn{1}{c|}{\cellcolor{green!75}M} & \multicolumn{1}{c|}{\cellcolor{green!75}M} & \multicolumn{1}{c|}{\cellcolor{green!75}M} & \multicolumn{1}{c|}{\cellcolor{green!75}M}  & \multicolumn{1}{c|}{\cellcolor{green!75}M} & \multicolumn{1}{c|}{\cellcolor{green!75}M} & \multicolumn{1}{c|}{\cellcolor{green!75}M} & \multicolumn{1}{c|}{\cellcolor{green!75}M} & \multicolumn{1}{c|}{\cellcolor{green!75}M} & \multicolumn{1}{c|}{\cellcolor{green!75}M} & \multicolumn{1}{c|}{\cellcolor{green!75}M} & \multicolumn{1}{c|}{\cellcolor{green!75}M} & \multicolumn{1}{c|}{\cellcolor{green!75}M} & \multicolumn{1}{c|}{\cellcolor{green!75}M} & \multicolumn{1}{c|}{\cellcolor{green!75}M} & \multicolumn{1}{c|}{\cellcolor{green!75}M} & \multicolumn{1}{c|}{\cellcolor{green!75}M} & \multicolumn{1}{c|}{\cellcolor{green!75}M} & \multicolumn{1}{c|}{\cellcolor{green!75}M} & \multicolumn{1}{c|}{\cellcolor{green!75}M} & \multicolumn{1}{c|}{\cellcolor{green!75}M} \\ \hline
\rotatebox[origin=c]{90}{Znak~}                        & \rotatebox[origin=c]{90}{Wykładnik~}                       &                        &                        &                        &                        &                        &                         &                        & \rotatebox[origin=c]{90}{Mantysa~}                       &                        &                        &                        &                        &                        &                         &                        &                        &                        &                       
\end{tabular}
\normalsize
\end{table}

Standard wyróżnia następujące precyzje:
\begin{itemize}
\item połowicza (16 bit),
\item pojedyncza (32 bit),
\item podwójna (64 bit),
\item podwójna rozszerzona (80 bit).
\end{itemize}

Przykład:
Liczba $17_{10}$ zapisana jako 32-bitowa liczba zmiennoprzecinkowa ma postać:

0|10000011|00010000000000000000000

czyli

\begin{itemize}
	\item{Znak -- 0 -> liczba dodatnia}
	\item{Wykładnik -- $10000011_2 == 2^{131 - 127}_{10} == 2^4_{10}$}
	\item{Mantysa -- $00010000000000000000000_2 == 1.0625_{10}$}
\end{itemize}

Po złożeniu wychodzi nam:
$(-1)^0 * 2^4 * 1.0625 == 1 * 16 * 1.0625 == 17.0$

Jaka jest główna zaleta takiego zapisu?
Ogromny wręcz zakres wartości możliwych do zapisania: korzystając z 32-bitowego typu zmiennoprzecinkowego możemy zapisać wartości z oszałamiającego zakresu <-3.40282e+38, 3.40282e+38>.

Powróćmy do pojęcia liczby znormalizowanej.

\subsubsection{Liczby znormalizowane i zdenormalizowane}

Standard IEEE754 wymaga, aby liczby były znormalizowane (oprócz wartości specjalnych) -- mantysa powinna być w zakresie $1 \le |M| < 2$. Liczbę taką można poznać po tym, że wykładnik jest \textbf{różny} od zera. Wtedy mantysa liczby jest postaci 1.bb...bb.

W wypadku, gdy wykładnik liczby jest równy zero, mówimy, że liczba jest zdenormalizowana, a jej mantysa ma postać 0.bb...bb.

Liczby zdenormalizowane są potrzebne do zapisania wartości tak bliskich 0, że ich ukryty bit w mantysie musi wynosić 0.

Będąc przy tym temacie warto wspomnieć o wartościach specjalnych.

\subsubsection{Wartości specjalne}


Liczby zmiennoprzecinkowe zawierają zarezerwowane wartości dla liczb specjalnych.

Wartości specjalne to:
\begin{itemize}
	\item{Zero -- oprócz bitu znaku zawiera same zera (tak więc mamy tu zero dodatnie i ujemne).}
	\item{$\pm\infty$ -- wszystkie bity wykładnika wynoszą 1, mantysa 0.}
	\item{Nie-liczba (NaN) -- wykładnik również przyjmie same jedynki, natomiast mantysę $\ne$ 0.}
	\item{Liczby zdenormalizowane posiadają w wykładniku same zera, natomiast mantysa jest postaci $0.bbb\ldots bb$.}

\end{itemize}
\begin{table}[H]
\centering
\caption{Liczby w standardzie IEEE754}
\begin{tabular}{|c|c|c|c|}
\hline
\textbf{Znak} & \textbf{Wykładnik} & \textbf{Mantysa} & \textbf{Wartość} \\ \hline
0 & $00\cdots00$ & $00\cdots00$ & $+0$ \\ \hline
0 & $00\cdots00$ & \begin{tabular}[c]{@{}c@{}}$00\cdots01$\\$\vdots$\\$11\cdots11$\end{tabular} & \begin{tabular}[c]{@{}c@{}}Dodatnia zdenormalizowana\\($0.f\times 2^{-N+1}$)\end{tabular} \\ \hline
0 & \begin{tabular}[c]{@{}c@{}}$00\cdots01$\\$\vdots$\\$11\cdots10$\end{tabular} & XX$\cdots$XX & \begin{tabular}[c]{@{}c@{}}Dodatnia znormalizowana\\($1.f\times 2^{E-N}$)\end{tabular} \\ \hline
0 & $11\cdots11$ & $00\cdots00$ & $+\infty$ \\ \hline
0 & $11\cdots11$ & \begin{tabular}[c]{@{}c@{}}$00\cdots01$\\$\vdots$\\$01\cdots11$\end{tabular} & SNaN \\ \hline
0 & $11\cdots11$ & 1X$\cdots$XX & QNaN \\ \hline
1 & $00\cdots00$ & $00\cdots00$ & $-0$ \\ \hline
1 & $00\cdots00$ & \begin{tabular}[c]{@{}c@{}}$00\cdots01$\\$\vdots$\\$11\cdots11$\end{tabular} & \begin{tabular}[c]{@{}c@{}}Ujemna zdenormalizowana\\($-0.f\times 2^{-N+1}$)\end{tabular} \\ \hline
1 & \begin{tabular}[c]{@{}c@{}}$00\cdots01$\\$\vdots$\\$11\cdots10$\end{tabular} & XX$\cdots$XX & \begin{tabular}[c]{@{}c@{}}Ujemna znormalizowana\\($-1.f\times 2^{E-N}$)\end{tabular} \\ \hline
1 & $11\cdots11$ & $00\cdots00$ & $-\infty$ \\ \hline
1 & $11\cdots11$ & \begin{tabular}[c]{@{}c@{}}$00\cdots01$\\$\vdots$\\$01\cdots11$\end{tabular} & SNaN \\ \hline
1 & $11\cdots11$ & 1X$\cdots$XX & QNaN \\ \hline
\end{tabular}
\end{table}
N -- obciążenie wykładnika, X -- dowolna wartość, QNaN -- quiet not-a-number (nie zgłaszają wyjątków), SNaN -- signaling not-a-number (zgłaszają wyjątki)

\begin{table}[H]
\centering
\caption{Specjalne operacje arytmetyczne}
\begin{tabular}{|c|c|}
\hline
\textbf{Operacja} & \textbf{Wynik} \\ \hline
$\frac{n}{\pm\infty}$ & 0  \\ \hline
$\pm\infty\times\pm\infty$ & $\pm\infty$  \\ \hline
$\frac{\pm X}{0}$ & $\pm\infty$  \\ \hline
$X\times\pm\infty$ & $\pm\infty$  \\ \hline
\begin{tabular}[c]{@{}c@{}}$\infty+\infty$\\$\infty- -\infty$\end{tabular} & $+\infty$  \\ \hline
\begin{tabular}[c]{@{}c@{}}$-\infty-\infty$\\$-\infty+ -\infty$\end{tabular} & $-\infty$  \\ \hline
\begin{tabular}[c]{@{}c@{}}$\infty-\infty$\\$-\infty+ \infty$\end{tabular} & NaN \\ \hline
$\frac{\pm 0}{\pm 0}$ & NaN \\ \hline
$\frac{\pm\infty}{\pm\infty}$ & NaN \\ \hline
$\infty\pm 0$ & NaN \\ \hline
\end{tabular}
\end{table}

Standard wyróżnia również wyjątki:
\begin{itemize}
\item invalid operation --- niewłaściwa operacja,
\item division by zero --- dzielenie przez zero,
\item overflow --- nadmiar -- liczba jest za duża,
\item underflow --- niedomiar -- utrata precyzji przy liczbach bliskich zero,
\item inexact --- niedokładność -- podczas operacji zaokrąglania.
\end{itemize}

\subsection{Działania}
\subsubsection{Arytmetyka stałoprzecinkowa}
\begin{itemize}
\item \textbf{Dodawanie/odejmowanie}
	\begin{itemize}
	\item dodawanie kolejnych cyfr, z zachowaniem ich pozycji
	\item występują przeniesienia
	\item odejmowanie -- dodawanie liczby przeciwnej (odjętej od zera)
	\end{itemize}
\item \textbf{Mnożenie}
	\begin{itemize}
	\item mnożenie mnożnej kolejnymi cyframi mnożnika, potem dodanie kolejnych sum częściowych
	\item w systemie uzupełnieniowym jest podobnie jak w naturalnym, gdy mnożnik ujemny wymagana jest korekcja wyniku
	\end{itemize}
\item \textbf{Dzielenie}
	\begin{itemize}
	\item wynik dzielenia jest przybliżony (nie ma gwarancji, że dla dzielnej $X$ oraz dzielnika $D$ istnieje liczba $Q$, że $X = Q * D$, można to skorygować dodając resztę z dzielenia)
	\item właśnie ogarnąłem, że opisałem fakty, których uczyliśmy się w podstawówce
	\end{itemize}
\end{itemize}

\subsubsection{Arytmetyka zmiennoprzecinkowa}
W przypadku liczb zmiennoprzecinkowych ważna jest kolejność działań!
Wynika to z błędów precyzji i zaokrągleń.

\begin{itemize}
\item \textbf{Dodawanie/odejmowanie}
	\begin{itemize}
	\item wymaga wyrównania wykładników (mniejszy do większego)
	\item denormalizacja jednej liczby, następnie dodanie/odjęcie mantys, potem normalizacja wyniku
	\item nadmiar/niedomiar może wystąpić przy normalizacji
	\end{itemize}
\item \textbf{Mnożenie/dzielenie}
	\begin{itemize}
	\item mnożenie -- przemnożenie mantys, dodanie wykładników, normalizacja
	\item dzielenie -- podzielenie mantys, odjęcie wykładników, normalizacja
	\item może wystąpić nadmiar/niedomiar
	\end{itemize}
\end{itemize}

\subsubsection{Zaokrąglanie}
Na koniec powiedzmy sobie o zaokrągleniach liczb.

Są one potrzebne, ponieważ w systemie binarnym nie zawsze da się dokładnie zapisać wszystkie liczby (ten problem występuje również w innych systemach).

Przykładowo, liczba dziesiętna $\frac{1}{10}_{10} == 0.1_{10}$ daje się łatwo zapisać w systemie dziesiętnym, jednak w binarnym jest ona ułamkiem okresowym $0.1(0011)_2$.
Ze względu na naturę komputerów, pamięć jest ograniczona, a co za tym idzie nie jesteśmy w stanie zapisać liczb z nieskończoną precyzją.

Kiedy komputer napotyka liczbę, której nie jest w stanie dokładnie zapisać, zaokrągla ją.

Zanim przejdziemy do omawiania konkretnych zaokrągleń, porozmawiajmy o trzech bitach, które są ukryte za mantysą: G,R,S.

Bity G(uard)R(ound)S(ticky) istnieją dla celów zaokrągleń i przechowywane są w nich bity niemieszczące się w mantysie. Już tłumaczę.

Jeśli chcemy naszą mantysę podzielić przez np. 8, czyli przesunąć bity w prawo o 3 pozycje, to wygląda to następująco:

mantysa |GRS
 
000111010|000

xxx000111|010

001000111|010

Skąd wzięło się 001 na miejscu xxx i czym w końcu są te bity GRS?

Bity 001 na początku wzięły się stąd, że mieliśmy do czynienia z liczbą znormalizowaną, więc na początku stała ukryta 1, a że mantysa zawsze jest dodatnia, to przed nią stoi nieskończenie wiele zer.
Jasne? Jasne.

Bity G oraz R to po prostu dwa dodatkowe bity używane przez implementację do przechowywania dodatkowych wartości.

Bit S mówi nam czy po prawej od bitów G oraz R może znajdować się jeszcze jakaś '1' -- jeśli jest możliwe, że wśród utraconych bitów była jakaś '1' to bit jest ustawiany na '1', a w przeciwnym wypadku na '0'.

Dobrze, rozpatrzmy różne zaokrąglenia na przykładzie liczby $0.1(0011)_2$

\textbf{Zaokrąglanie do zera}

Na początku wybieramy ile miejsca możemy poświęcić na zapisanie naszej liczby, a resztę kasujemy. Bitami GRS w ogóle się nie przejmujemy.
Wybierzmy 4 miejsca po przecinku.

$0.100110011001100110011...0011_2$ => $0.1001_2$

Już. Proste, prawda?

\textbf{Zaokrąglanie do +$\infty$}

Niezależnie od wartości, zawsze zaokrąglamy w górę.

TODO
dokonczyc zaokraglenia



\subsection{Podsumowanie}
\begin{table}[H]
\centering
\begin{tabular}{|l|l|l|}
\hline
\multicolumn{1}{|c|}{\textbf{Cecha}}                          & \multicolumn{1}{c|}{\textbf{L. stałoprzecinkowe}}                                               & \multicolumn{1}{c|}{\textbf{L.zmiennoprzecinkowe}}                                                             \\ \hline
Reprezentacja                                                 & \begin{tabular}[c]{@{}l@{}}Tak jak w systemie \\ pozycyjnym lub \\ uzupełnieniowym\end{tabular} & \begin{tabular}[c]{@{}l@{}}Reprezentacja przez:\\ znak, wykładnik, mantysę\end{tabular}                        \\ \hline
Dokładność                                                    & \begin{tabular}[c]{@{}l@{}}Mały zakres, \\ dokładne wyniki\end{tabular}                         & \begin{tabular}[c]{@{}l@{}}Duży zakres, możliwe \\ zaokrąglenia, utrata precyzji\end{tabular}                  \\ \hline
\begin{tabular}[c]{@{}l@{}}Obsługa \\ błędów\end{tabular}     & \begin{tabular}[c]{@{}l@{}}Może wystąpić \\ przepełnienie\end{tabular}                          & \begin{tabular}[c]{@{}l@{}}Zaokrąglenia, błędne operacje, \\ przepełnienie, niedomiar\end{tabular}             \\ \hline
\begin{tabular}[c]{@{}l@{}}Wartości \\ specjalne\end{tabular} & \multicolumn{1}{c|}{---}                                                                        & \begin{tabular}[c]{@{}l@{}}+0, -0, $+\infty$, $-\infty$,\\ QNaN, SNaN, liczby \\ zdenormalizowane\end{tabular} \\ \hline
Arytmetyka                                                    & \begin{tabular}[c]{@{}l@{}}Wszystkie twierdzenia \\ prawdziwe\end{tabular}                      & \begin{tabular}[c]{@{}l@{}}Kolejność działań \\ wpływa na wynik\end{tabular}                                   \\ \hline
\end{tabular}
\end{table}